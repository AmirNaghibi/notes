%
% BUS 314: Resourcing New Ventures - A Course Overview
% Section: Capitalization Structure
%
% Author: Jeffrey Leung
%

\section{Capitalization Structure}
	\label{sec:capitalization-structure}
\subsection{Introduction}
	\label{subsec:capitalization-structure:introduction}
\begin{easylist}

& \textbf{Capitalization structure:} Distribution of debt and equity in a company

& \textbf{Venture capital:} Private equity funding provided to startups at early funding rounds

& \textbf{Stock/share:} Equity in an entity

& \textbf{Vesting:} Gradual transfer of legal ownership of equity
	&& \textbf{Cliff:} Period of time until which no shares vest
	&& \textbf{Accelerated vesting:} Ability to increase the speed at which shares vest

& \textbf{Security:} Order of reimbursement of assets during liquidation
	&& In order of most security to least security:
		&&& Debtors
		&&& Preferred shareholders
			&&&& Series B shareholders
			&&&& Series A shareholders
		&&& Common shareholders

& \textbf{Pre-money valuation:} Valuation of a company before funding

& \textbf{Post-money valuation:} Valuation of a company after funding
	&& Equation: Pre-money valuation + value of funding

& \textbf{Valuation:} Perceived balance of risk and reward

& \textbf{Pro-rata ownership:} Ability to maintain a certain percentage of ownership of the company by being given the option to purchase more shares upon any dilution

& \textbf{Term sheet:} Formal document between an investor and investing company containing terms on which investment occurs
	&& May have legally binding provisions (e.g. confidentiality)
	&& Not legally binding; does not imply a closed deal

& \textbf{Key control terms:} Contract terms to manage risk
	&& E.g. Protective provisions, anti-dilution provisions
	&& \textbf{Drag-along rights:} Contract term where a majority of voters can override the minority
	&& Control provisions exist for voting, which can be majority or super majority

& \textbf{Investment memo:} Confidential internal documentation about a decision on a potential investment

\end{easylist}
\subsection{Common Stock}
	\label{subsec:capitalization-structure:common-stock}
\begin{easylist}

& \textbf{Common stock:} Stock which has claim to assets/dividends, and gives voting rights

& Does not have a participation cap

& Calculation of common stock liquidation:
\end{easylist}
\begin{align*}
	\textrm{Common stock liquidation }
	& = \textrm{ Percent of ownership } \times \\
	& ( \textrm{Total liquidated } - \textrm{ Preferred stock liquidation} )
\end{align*}
\begin{easylist}

\end{easylist}
\subsection{Preferred Stock}
	\label{subsec:capitalization-structure:types-of-preferred-stock}
\begin{easylist}

& \textbf{Preferred stock:} Stock which has higher claim to assets/dividends than common stock, but has no voting rights
	&& Convertible to common stock at a 1-to-1 ratio

& \textbf{Series A round:} First significant round of venture capital financing
	&& \textbf{Series A stock:} Preferred stock generated during the Series A round

& \textbf{Series B round:} Second significant round of venture capital financing
	&& \textbf{Series B stock:} Preferred stock generated during the Series A round

& Investors can follow the rules for preferred stock, or convert it to common stock which removes the rules

& \textbf{Liquidation preference:} Term on preferred stock which provides guaranteed investor protection by returning liquidated proceeds first if a venture fails
	&& Calculated as a multiple of the investment
	&& Above 1x is unreasonable
	&& E.g. Given an investment of \$1M, a liqudation preference of 2x, and a liquidation of \$5M (at least greater than \$2M), the investor is guaranteed a return of \$2M

& \textbf{Participation cap:} Maximum amount of liquidated proceeds received from preferred stock, relative to the investment
	&& E.g. Given an investment of \$1M and a participation cap of 3x, the maximum amount received upon liquidating the preferred stock and participating is \$3M

& \textbf{Participating (preferred) stock:} Preferred stock which receives liquidated proceeds (investment times liquidation preference) first, then continues to receive a percentage of common stock

	&& Calculation of participating preferred stock liquidation:
\end{easylist}
\begin{align*}
	\textrm{Participating preferred liquidation } =
	& \textrm{ } MAX( \\
	& \qquad MIN( \\
	& \qquad \qquad \textrm{Investment } \times \textrm{ LiquidPref } + \\
	& \qquad \qquad \textrm{ParticipatingPrefStockPercent } \times \textrm{ TotalLiquidated}, \\
	& \qquad \qquad \textrm{Investment } \times \textrm{PartCap} \\
	& \qquad ), \\
	& \qquad toCommon(\textrm{PrefStock}) \\
	& )
\end{align*}
\begin{easylist}

& \textbf{Non-participating (preferred) stock:} Preferred stock which receives liquidated proceeds (investment times liquidation preference) but does not receive a percentage of common stock

	&& Calculation of non-participating preferred stock liquidation:
\end{easylist}
\begin{align*}
	\textrm{Non-participating preferred liquidation } =
	& \textrm{ } MAX( \\
	& \qquad MIN( \\
	& \qquad \qquad \textrm{Investment } \times \textrm{ LiquidPref }, \\
	& \qquad \qquad \textrm{Investment } \times \textrm{PartCap} \\
	& \qquad ), \\
	& \qquad toCommon(\textrm{PrefStock}) \\
	& )
\end{align*}
\begin{easylist}

& Steps to calculate exit value for a given investor with preferred shares:
	&& Calculate the value of liquidating common shares:
		&&& Change the class of shares from preferred to common, keeping the percentage the same
		&&& Multiply the percentage of common share by the liquidation amount
	&& Calculate the value of liquidating preferred shares:
		&&& Take the original investment
		&&& Multiply it by the liquidation preference
		&&& Cap it by the liquidation amount
		&&& If participating, add the participating value:
			&&&& Take the liquidation amount
			&&&& Remove the preferred share liquidation
			&&&& Multiply it by the percentage of preferred shares
			&&&& Add this to the preferred share liquidation amount calculated above
		&&& Apply the participation cap:
			&&&& Multiply the original investment by the participation cap
			&&&& Cap the liquidation amount above by this amount
		&&& Take the larger value of this vs. the value of liquidating common shares

& \textbf{Participation right:} Term where investors receive additional returns according to cap table percentage upon liquidation

& Avoid:
	&& \textbf{Redemption right:} Term which provides investors with the ability to withdraw the investment
	&& Milestone/tranching-based financing: Ability to hold back investments until a certain date or milestone
	&& Reciprocal contract

\end{easylist}
\subsection{Stock Options}
	\label{subsec:capitalization-structure:stock-options}
\begin{easylist}

& \textbf{Stock option:} Right to buy common stock (at a discounted rate)
	&& Rate to purchase is set when first issued
	&& Not stock/shares, so they cannot vote
	&& Can be converted to common shares
	&& Will vest over time

& Stages in the capitalization table:
	&& Reserved
	&& Allocated
	&& Vested
	&& Exercised

& \textbf{Exercise/strike price:} Discounted price at which a share can be bought using a stock option

\end{easylist}
\subsection{Stages}
	\label{subsec:capitalization-structure:stages}
\begin{easylist}

& \textbf{Startup:} Venture which has not yet received Series B funding
	&& Financing from equity (initial investors), customers, employees (who will work for equity and lower salary)

& \textbf{Scaling:} Venture which has received Series B funding but is not yet public or having private equity
	&& Financing from equity, debt, and customers

& \textbf{Mature:} Venture which is stable, being public or having private equity
	&& Financing from equity, debt, and customers

\end{easylist}
\clearpage
