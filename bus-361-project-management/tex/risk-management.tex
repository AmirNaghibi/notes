%
% BUS 361: Project Management - A Course Overview
% Section: Risk Management
%
% Author: Jeffrey Leung
%

\section{Risk Management}
	\label{sec:risk-management}
\begin{easylist}

& \textbf{Risk:} Uncertain event or condition which affects a project objective positively or negatively
	&& Often occurs when assumptions are made
	&& Types of risk: Financial, technical, commercial success, execution, contractual/legal

& \textbf{Risk management:} Identification, analysis, response to, and monitoring of risk factors
	&& Maximization of positive events, and minimizing likelihood and consequences of negative events

& Methods of identifying risk:
	&& WBS analysis
	&& Reviews of scope, stakeholders, and documents
	&& SWOT analysis
	&& Interviews and research

& Process of assessing risk:
	&& Identify probability of occurrence and potential consequences (both on a scale of Low, Guarded, Moderate, High, or Extreme)
	&& Equation: Event risk = Probability $\times$ Consequences
	&& Subjective
	&& \textbf{Probability/Likelihood Impact Matrix:} Organizational tool to graph the likelihood and consequences of risks for prioritization and comparison

& Responses to risk:
	&& \textbf{Avoidance:} Eliminating or limiting a risk through modifying limitations
	&& \textbf{Mitigation:} Eliminating or limiting a risk through limiting the probability or impact of a risk (e.g. simplifying processes, adding tests)
	&& \textbf{Transfer:} Eliminating or limiting a risk through shifting ownership or responsibility of the risk to another entity (e.g. warranties, contracts with fixed cost pricing)
	&& \textbf{Acceptance:} Eliminating or limiting a risk through being ready for the consequences (e.g. contingencies, fall-back plans, and workarounds)
	&& May alter WBS, network diagram, budget, scope, contingency reserves, etc.

& Monitoring risk:
	&& \textbf{Risk register:} Document which tracks risks, analyses, and response plans
	&& Monitor and report regularly (at least once per month)
	&& Stay updated on timelines for monitoring risks
	&& Track higher risks more frequently/closely

\end{easylist}
\clearpage
