%
% BUS 361: Project Management - A Course Overview
% Section: Cost and Resource Management
%
% Author: Jeffrey Leung
%

\section{Cost and Resource Management}
	\label{sec:cost-resource-management}
\begin{easylist}

& \textbf{Effort:} Actual time invested in an activity (e.g. man hours)
& \textbf{Duration:} Time between activity start and finish

& Process to calculate resource cost:
	&& Create a WBS document
	&& Effort:
		&&& Create an estimate for total effort of all work packages
		&&& Multiply effort by resource cost
	&& Cost:
		&&& Create an estimate for total cost of all work packages
		&&& Apply contingency to effort and cost

& Types of costs:
	&& \textbf{Direct:} Costs which are clearly assigned to the output
		&&& E.g. Labor, materials, subcontractors, equipment, facilities, travel
	&& \textbf{Indirect:} Costs from internal spending which indirectly translate to output
		&&& E.g. Overhead costs (utilities, taxes, insurance, maintenance, depreciation) or administration (advertising, salaries, sales, commissions)
	&& \textbf{Fully loaded rate:} Labor costs which are calculated using an overhead multiplier
	&& \textbf{Nonrecurring:} Charges which are applied once (e.g. preliminary analyses, training)
	&& \textbf{Recurring:} Charges which continue over the timeline (e.g. labor, material)
	&& \textbf{Fixed:} Charges which do not change with usage (e.g. leasing capital hardware)
	&& \textbf{Variable:} Charges which increase with usage (e.g. equipment degradation)
	&& \textbf{Normal:} Charges which are expected and agreed upon
	&& \textbf{Expedited:} Charges which are unplanned and occur to speed up completion

& \textbf{Gantt chart:} Diagram of project schedule with start and finish dates of summary elements
	&& Simple to create and read
	&& Can be used for control
	&& Does not display details, sequencing, path to completion
	&& Does not provide information on efficient resources usage

\end{easylist}
\subsection{Critical Path Method}
	\label{subsec:critical-path-method}
\begin{easylist}

& \textbf{Float/slack:} Amount of time an activity can be delayed without affecting the project
	&& \textbf{Free float:} Amount of time an activity can be delayed without affecting the following activity
	&& \textbf{Total float:} Amount of time an activity can be delayed without affecting project completion date

& \textbf{Critical path:} Sequence of activities which determines the shortest total duration of the project
	&& Given possible sequences of precedence activities, the longest path has no float and is the critical path

& Critical path method:
	&& Conduct a forward pass to determine earliest start/end activity times
	&& Conduct a backward pass to determine latest start/end activity times
	&& Calculate the possible slack for each item

& To shorten the critical path:
	&& Eliminate tasks
	&& \textbf{Crashing:} Speeding up a task to reduce project duration
		&&& Shorten all/early/long/easy tasks, or tasks which cost less to speed up
	&& \textbf{Fast tracking:} Allow parallel work
	&& Overlap sequential tasks

& Process to create a schedule:
	&& Using the effort calculated, create duration estimates
	&& Create a network diagram
	&& Generate a critical path from the network diagram
	&& Take the total duration from the critical path

\end{easylist}
\clearpage
