%
%% Author: Jeffrey Leung
%% Last edited: 2015-07-05
%%
%% This document contains section 2 (digital systems) of a course overview of CMPT 150.
%

\section{Digital Systems}

\begin{easylist}[itemize]

& \emph{Digital system:} Electronic circuit which processes discrete signals representing logic values
& Computer design:
	&& Instruction set architecture involves the selection, design, and representation of a set of instructions and of some basic data types
	&& Construction of a circuit that can:
		&&& Interpret binary sequences representing instructions
		&&& Perform computations on binary sequences as directed by instructions
		&&& Express results as binary encoded data
		
& \emph{Register:} Component which stores one binary sequence
	&& \emph{Memory:} A 1-dimensional array of registers
& \emph{Bus:} Component which transmits one binary sequence
	&& \emph{Signal line:} A bus of size 1
	
& \emph{Registry Transfer Notation:} Convention for naming registers
	&& Register names begin with an uppercase letter \\
	   Bus names begin with a lowercase letter
	&& Bit positions are specified by a number in parenthesis after the name
		&&& E.g. RNG(MSB), abc(12)
	&& \emph{Field:} Sequence of bits within a binary sequence written as \\ NAME(start:end) where the bit positions include the start and end positions
		&&& E.g. For INST = 0110 1101 0100 0000, \\
		INST(15:8) = 0110 1101
		&&& E.g. For $R = 10010.011_{2}$, \\
		R(INT) = R(8:3) and R(FRAC) = R(2:0)
		
%TODO CMPT 150 lecture notes, page 6

\end{easylist}