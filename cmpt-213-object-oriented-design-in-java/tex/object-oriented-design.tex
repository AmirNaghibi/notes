%
% CMPT 213: Object Oriented Design in Java - A Course Overview
% Section: Object-Oriented Design
%
% Author: Jeffrey Leung
%

\section{Object-Oriented Design}
	\label{sec:object-oriented-design}
\begin{easylist}

& Effective because classes represent stable problem domain concepts
& Design involves identifying classes, responsibilities, and relationships

& \textbf{Object:} Unique software entity with state and behaviours
	&& \textbf{State:} Information about an object
	&& \textbf{Behaviour:} Methods and operations which view and/or modify state
	&& \textbf{Class:} Type of a set of objects with the same behaviours and set of possible states
		&&& \textbf{Final class:} Class which cannot be inherited from
	&& \textbf{Field/instance data:} Member variable stored by an object
	&& \textbf{Method:} Member function of a class
		&&& \textbf{Final method:} Method which cannot be overridden
	&& Statics:
		&&& \textbf{Static/class field:} Field for which only a single instance exists and is shared between all classes
		&&& \textbf{Static/class method:} Method which can be called on the class without a constructed object
	&& Unique types of objects:
		&&& Agent: Object which performs a specific task
		&&& Users/roles, events, systems, interfaces

& Anonymous classes:
	&& \textbf{Anonymous class:} Single-use implementation of an interface, defined only once within another body of code
	&& \textbf{Anonymous object:} Instance of an unnamed class

& Relationships between objects:
	&& \textbf{Dependency (uses):} Relationship between two classes where one uses the other (i.e. the first may need to change if the second changes)
		&&& \textbf{Coupling:} Relationship between two classes where one is a dependency of the other
			&&&& The greater the coupling, the greater the changes required when one component is modified
			&&&& Ideally minimized
	&& \textbf{Aggregation (has a):} Relationship between two classes where one contains the other
	&& \textbf{Inheritance (is a):} Relationship between two classes where one is a subclass of the other (see subsection~\ref{subsec:object-oriented-design:inheritance})

& \textbf{Side effect:} Observable change to state caused by code execution

& \textbf{Idiom:} Common practice

& \textbf{Protected:} Access class which is available at package-level as well as derived classes, but not external code
	&& Breaks encapsulation; exposes implementation details to derived classes
	&& Difficult because no control over the extending clases

\end{easylist}
\subsection{Public Interface}
	\label{subsec:object-oriented-design:public-interface}
\begin{easylist}

& \textbf{Interface:} Set of methods which an implementing class can choose to implement

& Object type cannot be dynamically changed

\end{easylist}
\subsection{Principles}
	\label{subsec:object-oriented-design:principles}
\begin{easylist}

& \textbf{Encapsulation:} Characteristic of a module which only exposes some public interfaces, allowing some internal changes to be made without breaking outward interface
	&& Modules manage their own state
	&& Reduces scope of change (which encourages modification) and cognitive load

& \textbf{Immutability:} Characteristic of an entity which has no methods which change its visible state
	&& A modification to an object must return a new object
	&& An accessor should return a value or immutable reference
	&& Avoids issues from partial setters and shared references
	&& \textbf{Final:} Keyword which declares a field to have an immutable reference

& \textbf{Command-Query Separation:} Principle which states that methods should not be both mutator and accessor, to avoid unexpected side effects
	&& \textbf{Command:} Mutator method
		&&& Should return null
	&& \textbf{Query:} Accessor method

& \textbf{Responsibility heuristic:} Tendency to avoid exposing internal details only for external access

\end{easylist}
\subsection{SOLID Principles}
	\label{subsec:object-oriented-design:solid-principles}
\begin{easylist}

& \textbf{Single Responsibility Principle:} Each class only has one responsibility

& \textbf{Open-Closed Principle:} Classes must be open for extension but closed for modification

& \textbf{Liskov Substitution Principle:} Subtype objects must be interchangeable with base objects

& \textbf{Interface Segregation Principle:} Models should support multiple possible interfaces

& \textbf{Dependency Inversion Principle:} Code should depend on abstractions, not concrete classes

\end{easylist}
\subsection{Polymorphism}
	\label{subsec:object-oriented-design:polymorphism}
\begin{easylist}

& \textbf{Concrete type:} Exact instantiated type of an object

& \textbf{Interface:} Definition of a set of public methods to be implemented
	&& A class implements an interface
	&& Code to an interface, not a concrete type
	&& Can inherit from another object

& \textbf{Polymorphism:} Characteristic of an object which can be one of multiple different concrete types
	&& Possible due to composition
	&& Allows late binding

& \textbf{Late binding:} Characteristic of a variable which is assigned a specific concrete type at runtime
	&& Allows loose coupling

\end{easylist}
\subsection{Inheritance}
	\label{subsec:object-oriented-design:inheritance}
\begin{easylist}

& \textbf{Inheritance:} Subclass created from and extending a superclass
	&& Allows sharing members (properties and methods)
	&& Allows polymorphism between two concrete classes
	&& \textbf{Single inheritance:} A subclass may have at most 1 superclass
	&& \textbf{Multiple inheritance:} A subclass may have multiple superclasses
	&& \textbf{Liskov Substitution Principle (LSP):} Rule which states that class B can inherit from A if and only if, for each method in A, the corresponding method in B accepts the same parameters (or more) and executes the same operations (or more)
		&&& I.e. Client code using the base class must be able to use the derived class without modification
		&&& Is-a relationships must satisfy the Liskov Substitution Principle
	&& Should be used for permanent relationships, not temporary assignments
	&& Favor polymorphism over inheritance

& \textbf{Superclass/Base class:} Class from which another class inherits
	&& Cannot be assigned to an instance of a subclass
	&& Referred to by \lstinline!super!
  && Cannot be final

& \textbf{Subclass/Derived class:} Class which inherits from another
	&& Can access non-private members of the superclass
	&& Cannot directly access private members of the superclass
	&& Can be assigned to an instance of a superclass

& \textbf{Constructor chaining:} Call to a superclass constructor by a constructor in the subclass
	&& Ensures base classes are initialized first
	&& In Java, if no superclass constructor is called, then superclass default 0-argument constructor is  automatically called

& Abstracts:
	&& \textbf{Abstract method:} Un-implemented method which must be implemented by any concrete subclass
	&& \textbf{Abstract class:} Class with abstract methods
		&&& Concrete derived classes must implement all abstract methods
	&& Abstract classes vs. interfaces:
		&&& Both require implementation of a given set of methods
		&&& Abstract classes can have member variables and implemented methods; interfaces cannot
		&&& A class can implement multiple interfaces; a class can only have a single (abstract) superclass

\end{easylist}
\subsection{Overriding}
	\label{subsec:object-oriented-design:overriding}
\begin{easylist}

& The overriding method:
	&& Must have identical signature (except for visibility and return type)
	&& Visibility cannot be reduced
	&& Cannot throw additional checked exceptions
	&& Can return a subclass of the original return type

& Method to override:
	&& Cannot be private, static, final

& \textbf{Shadow variable:} Subclass variable which overrides a variable of the superclass

\end{easylist}
\clearpage
