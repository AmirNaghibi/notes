%
% CMPT 276: Introduction to Software Engineering - A Course Overview
% Section: C++ REST SDK
%
% Author: Jeffrey Leung
%

\section{C++ REST SDK}
	\label{sec:cpp-rest-sdk}
\subsection{Overview}
	\label{subsec:cpp-rest-sdk:overview}
\begin{easylist}

	& \href{https://github.com/Microsoft/cpprestsdk}{C++ REST SDK}: Microsoft library which allows for client-server asynchronous communication
	
	& \href{https://github.com/Azure/azure-storage-cpp}{Microsoft Azure Storage Library}: Microsoft library which interfaces with Microsoft Azure Storage Tables
		&& \emph{Microsoft Azure Storage Tables:} Key-value cloud database
			&&& \emph{Key-value store:} Database consisting of a collection of arbitrary keys which are mapped to data values
			&&& Each key in the database consists of a partition name and a row name, and the entity mapped to by a key contains the values for the specific row
				&&&& Values for the specific row are a set of properties, each of which consists of a name and value

	& \emph{Uniform Resource Identifier (URI):} Name of a specific resource on a network
		&& URL path structure:
			&&& Consists of:
			\textit{protocol://address:port/parameter/parameter[/...]}
				&&&& E.g. \\
\textit{http://localhost:34568/Operation/TableName/PartitionName/RowName} \\
\textit{http://localhost:34568/Operation/TableName/AuthenticationToken\\/PartitionName/RowName}
			&&& \textbf{web::http::uri::decode()} converts from HTTP encoding to readable format
			&&& \textbf{web::http::uri::split\_path()} converts from readable format to a vector of strings containing the necessary information
	
\end{easylist}
\subsection{HTTP Requests}
	\label{subsec:cpp-rest-sdk:http-requests}
\begin{easylist}

	& HTTP request is represented by the class \textbf{web::http::http\_request}
		&& Components:
			&&& HTTP method: \textbf{http\_request.method()}
			&&& URI: \textbf{http\_request.relative\_uri().path()}
		&& Alteration methods:
			&&& Decoding URI from HTTP to internal format: \textbf{utility::string\_t web::http::uri::decode()}
			&&& Splitting paths at \textbf{/} characters: \textbf{utility::string\_t web::http::uri::split\_path()}
		&& Reply method: \textbf{http\_request.reply(http::status\_code)}
			&&& A JSON body may also be returned: \\
			\href{http://microsoft.github.io/cpprestsdk/classweb_1_1http_1_1http__request.html#a31d1463151d1b65f150ae03b2dca447f}
			{
				\textbf{http\_request.reply(http::status\_code, web::json::value)}
			}
				&&&& \href{http://microsoft.github.io/cpprestsdk/classweb_1_1json_1_1value.html#ad2073edb60952ef04d352906114c2b70}
				{
					Documentation for creating a JSON body
				}

\end{easylist}
