%
% CMPT 300: Operating Systems I - A Course Overview
% Section: Virtual Memory
%
% Author: Jeffrey Leung
%

\section{Virtual Memory}
	\label{sec:virtual-memory}
\begin{easylist}

& \textbf{Virtual memory:} Abstraction of memory management where logical memory can be conceptually larger than the physical address space
	&& Only part of a program is in memory at any given time

& \textbf{Demand paging:} Making pages available in virtual memory only as needed
	&& Decreases I/O requirements and waiting time, and increases number of potential concurrent processes
	&& Indicated by valid-invalid bit in page table (if invalid, the memory access is either illegal or not yet in memory)

\end{easylist}
\subsection{Page Faults}
	\label{subsec:virtual-memory:page-faults}
\begin{easylist}

& \textbf{Page fault:} Legal memory access of a page which is not currently in memory which triggers retrieval
	&& \textbf{Major page fault:} Page is brought from disk
	&& \textbf{Minor page fault:} Page table is updated (e.g. linking shared pages)
	&& If no free pages are available, a page will be removed from memory to make room
	&& To preserve atomicity of instructions which modify data, previous register values are saved and hardware accesses both ends of the data before execution to trigger page faults
	&& Effective Access Time:
	\end{easylist}
	\begin{align*}
		\textrm{Let } p &= \textrm{ page fault rate, } 0 \leq p \leq 1 \\
		& \textrm{ where 0 is no faults, 1 is every reference is a fault} \\
		\textrm{Let } t_m &= \textrm{ memory access time} \\
		\textrm{Let } t_p &= \textrm{ page fault time} \\
		& \textrm{(including interrupt, page read, restart)} \\
		\textrm{Effective access time }
		&= p \times t_p + (1 - p) \times t_m \\
	\end{align*}
	\begin{easylist}

		&&& Example:
		\end{easylist}
		\begin{align*}
			p &= 0.001 \textrm{ 1 fault per 1,000 memory accesses} \\
			t_m &= 200 \textrm{ ns} \\
			t_p &= 8 \textrm{ ms} = 8,000,000 \textrm{ ns}\\
			\textrm{Effective access time }
			&= p \times t_p + (1 - p) \times t_m \\
			&= 0.001 \times 8,000,000 + (1 - 0.001) \times 200 \\
			&= 8,000 + 199.8 \\
			&= 8,200 \textrm{ ns} \\
			\textrm{Slowdown } &= \frac{8200 \textrm{ ns}}{200 \textrm{ ns}} = 40 \textrm{ times slower}
		\end{align*}
		\begin{easylist}
		
	&& Process:
		&&& Send a trap to the OS
		&&& Save the user registers and process states
		&&& Determine that the interrupt was a page fault
		&&& Check legality of the page access and determine the location of the page
		&&& Issue a read from the disk to a free frame
		&&& Wait until the read request is serviced, with seek and/or latency time
		&&& Transfer the page to a free frame
		&&& While waiting, allocate the CPU to some other process
		&&& Receive a completion interrupt from the disk I/O subsystem
		&&& Save the registers and process state for the other process
		&&& Determine that the interrupt was from the disk
		&&& Correct the page tables to show page is now in memory
		&&& Wait for the CPU to be allocated to this process again
		&&& Restore the user registers, process state, and new page table
		&&& Resume the interrupted instruction

& \textbf{Copy-on-Write (COW):} Method of quickly/efficiently creating a child process by sharing the pages of the parent process in memory and copying the data only when a shared page is modified

& \textbf{Pre-paging:} Bringing pages into memory before they are referenced to decrease page faults
	&& May waste memory and I/O if the page is not used

\end{easylist}
\subsection{Page Replacement}
	\label{subsec:virtual-memory:page-replacement}
\begin{easylist}

& \textbf{Page replacement algorithm:} Method of selecting a victim page to be removed from memory which is designed to minimize fault rate
	&& Dirty/modified bit in page table entries to save swap-out overhead if the victim page was not modified
	&& \textbf{I/O interlock:} Page currently used for I/O is locked in memory and cannot be replaced during a fault

& \textbf{First-In-First-Out (FIFO) page replacement algorithm:} Removing from memory the oldest page upon a fault
	&& Could consistently remove a heavily-used page
	&& \textbf{Bélády's anomaly:} Phenomenon in the FIFO page replacement algorithm where increasing available frames may potentially result in greater page faults
	
& \textbf{Least Recently Used (LRU) page replacement algorithm:} Removing from memory the page which has not been used for the longest period, upon a page fault
	&& Counter tracking: Using a data field to store CPU clock of the last memory access and searching for the oldest page
		&&& Search is time-consuming, many memory writes, potential clock overflow, requires hardware
	&& Stack tracking: Using a stack created from a doubly-linked list for which each page access moves the page to the top
		&&& Memory reference updating is expensive, requires hardware
		&&& No search is required

& \textbf{Second-chance/clock replacement:} Using a set of reference bits which are set when a page is referenced and reset when a page is replaced, and removing a page which was not used since the last replacement

& \textbf{Least Frequently Used (LRU) page replacement algorithm:} Removing from memory the page which has been used the least
	&& Removes pages which are not used often
	&& Early heavily-used pages will stay in memory
& \textbf{Most Frequently Used (MRU) page replacement algorithm:} Removing from memory the page which has been used the most
	&& Pages with smaller counts are estimated to be newer and kept
	&& Heavily-used pages may be removed more frequently

& \textbf{Optimal page replacement algorithm:} Removing from memory the page which will not be used for the longest period of time, upon a fault
	&& Only possible in theory or when analyzing a past set of page accesses
	&& Used to benchmark existing algorithms

\end{easylist}
\subsection{Frame Allocation}
	\label{subsec:virtual-memory:frame-allocation}
\begin{easylist}

& \textbf{Equal allocation:}

& \textbf{Proportional allocation:} Allocating frames per process relative to the size of the process
	&& Formula:
	\end{easylist}
	\begin{align*}
		\textrm{Let } m &= \textrm{ total number of frames} \\
		\textrm{Let } p_i & \textrm{ be a given process} \\
		\textrm{Let } s_i &= \textrm{ size of process } p_i \\
		\textrm{Page allocation } a_i \textrm{ for } p_i
		&= \frac{s_i}{\sum s_i} \times m
	\end{align*}
	\begin{easylist}
	
		&&& Example:
		\end{easylist}
		\begin{align*}
			m &= 64 \textrm{ frames} \\
			s_1 \textrm{ for process } p_1 &= 10 \\
			s_2 \textrm{ for process } p_2 &= 127 \\
			\textrm{Page allocation } a_1
			&= \frac{s_1}{\sum s_1} \times m \\
			&= \frac{10}{10+127} \times 64 \\
			&\approx 5
		\end{align*}
		\begin{easylist}

& \textbf{Priority allocation:} Allocating frames per process relative to the priority of the process

& \textbf{Global replacement:}
& \textbf{Local replacement:}

& \textbf{Locality model:} Keeping in memory the pages necessary for the current segment of code

& \textbf{Working set:}
	&& Managed with an interval timer and set of reference bits (per page in memory), which is set when a page is referenced

& \textbf{Thrashing:} Situation where a process is starved of necessary frame allocations and must constantly swap pages

\end{easylist}
\subsection{Kernel Memory Allocation}
	\label{subsec:virtual-memory:kernel-memory-allocation}
\begin{easylist}

& \textbf{Buddy system:} Allocating memory from fixed-size contiguous segments with size to the power of 2
	&& May have up to 50\% internal fragmentation
& \textbf{Slab allocator:} Allocating memory from precreated spaces for kernel data structures
	&& \textbf{Cache:} Paged instantiation space in memory for an object
	&& \textbf{Slab:} Contiguous memory which contains multiple caches
	&& Memory allocation is fast and has no fragmentation

& \textbf{Memory mapping:} Mapping a file to a memory address space so modifications have the behaviour of ordinary memory accesses

\end{easylist}
\clearpage