%
% CMPT 310: Artificial Intelligence - A Course Overview
%
% Author: Jeffrey Leung
%

\documentclass[10pt, oneside, letterpaper, titlepage]{article}

\usepackage{amsfonts}
\usepackage{amsmath}
\usepackage[skip=10pt]{caption}
\usepackage[ampersand]{easylist}
	\ListProperties(
		Progressive*=5ex,
		Space=5pt,
		Space*=5pt,
		Style1*=\textbullet\ \ ,
		Style2*=\begin{normalfont}\begin{bfseries}\textendash\end{bfseries}\end{normalfont} \ \ ,
		Style3*=\textasteriskcentered\ \ ,
		Style4*=\begin{normalfont}\begin{bfseries}\textperiodcentered\end{bfseries}\end{normalfont}\ \ ,
		Style5*=\textbullet\ \ ,
		Style6*=\begin{normalfont}\begin{bfseries}\textendash\end{bfseries}\end{normalfont}\ \ ,
		Style7*=\textasteriskcentered\ \ ,
		Style8*=\begin{normalfont}\begin{bfseries}\textperiodcentered\end{bfseries}\end{normalfont}\ \ ,
		Hide1=1,
		Hide2=2,
		Hide3=3,
		Hide4=4,
		Hide5=5,
		Hide6=6,
		Hide7=7,
		Hide8=8 )
\usepackage{fancyvrb}
\usepackage{forest}
	\useforestlibrary{edges}
\usepackage{geometry}
	\geometry{margin=1.2in}
\usepackage{graphicx}
	\graphicspath{ {img/} }
\usepackage[colorlinks=true, linkcolor=blue]{hyperref}
\usepackage{listings}
\usepackage{lmodern} % Allows the use of symbols in font size 10; http://ctan.org/pkg/lm
\usepackage{mathtools}
\usepackage{pdflscape}
\usepackage{tabularx}
\usepackage{textcomp} % Allows the use of \textbullet with the font
\usepackage{tikz}
	\usetikzlibrary{positioning,shapes,arrows}
\usepackage{verbatim}

\renewcommand{\arraystretch}{1.4}
\renewcommand{\familydefault}{\sfdefault}

\title{CMPT 310: Artificial Intelligence \\\medskip \Large A Course Overview}
\author{Jeffrey Leung \\ Simon Fraser University}
\date{Spring 2019}

\begin{document}

	\maketitle
	\tableofcontents
	\clearpage

	%
% CMPT 213: Object Oriented Design in Java - A Course Overview
% Section: Introduction
%
% Author: Jeffrey Leung
%

\section{Introduction}
	\label{sec:introduction}
\begin{easylist}

& Standards:
	&& Make fields private when possible

& Commenting:
	&& Comment purpose of a class
	&& Name fields/methods/parameters so comments are unnecessary

& When possible, convert strings to non-string types internally for consistency

& \textbf{Clean code:} Code which is correct, easy to read/maintain, and conforms to a standard

& Software design:
	&& 4 steps:
		&&& Requirements
		&&& Design and implementation
		&&& Verification
		&&& Evolution
	&& Designing involves identifying classes, responsibilities, and relationships to create a diagram
	&& Implementation process options:
		&&& \textbf{Skeleton code:} Beginning minimal parts/features of a system
		&&& \textbf{Component-wise:} Creating components one at a time
	&& Methods of integrating code from multiple people:
		&&& \textbf{Continual integration:} Gradual system growth by constantly integrating changes
		&&& \textbf{Big Bang integration:} Building all parts separately without integrating until the end

& \textbf{Feature envy:} Characteristic of a class which relies heavily on another class
& Warning sign: Characteristic of a method which operates more strongly on another object than its own
& \textbf{Deprecation:} State where a public interface is no longer supported or recommended, and is slated to be removed in the future


& \textbf{try-catch:} Structure which watches for an exception and handles it
	&& Only one exception can be live at a given time
	&& \textbf{finally clause:} Optional clause after catch clauses which is executed regardless of the result
		&&& If exception is thrown, the finally clause is executed immediately afterwards
	&& \textbf{try-with-resources:} Block which cleans up a resource when a try block exits

& Exception: Issue which may be fixable and is not out of the software's control
	&& \textbf{Checked exception:} Exception which must be caught or listed in a throws clause
	&& \textbf{Unchecked exception:} Exception which will automatically propagate and does not require catching
		&&& E.g. RuntimeException
		&&& Preferred as it does not require modification of methods between try/catch, which decouples code

\end{easylist}
\clearpage

	%
% CMPT 310: Artificial Intelligence - A Course Overview
% Section: Problem Solving
%
% Author: Jeffrey Leung
%

\section{Problem Solving}
	\label{sec:problem-solving}
\begin{easylist}

& Environment: Observable, deterministic, discrete
& \textbf{State}: Possible arrangement of entities and agents in the environment
	&& Transition: Change from an initial state to another state
		&&& Successor function: Mapping from the current state and action to a new state
			&&&& Notation: $(\textrm{state}, \textrm{action}) \rightarrow \textrm{state}$
		&&& Goal state: Desired end state
		&&& Goal test: Condition which determines whether or not the current state is the goal state
			&&&& Notation: $(\textrm{state}) \rightarrow \textrm{true/false}$
		&&& Path cost: Numeric cost to take a path
	 	&&& Includes the possibility of no change
	&& E.g. For a two-square grid of a possibly dirty carpet with a vacuum cleaner, there are 4 possibilities for the dirty squares and 2 possibilities for the vacuum cleaner

\end{easylist}
\clearpage

	%
% CMPT 310: Artificial Intelligence - A Course Overview
% Section: Search Algorithms
%
% Author: Jeffrey Leung
%

\section{Search Algorithms}
	\label{sec:search-algorithms}
\subsection{Uninformed Algorithms}
	\label{subsec:uninformed-algorithms}
\begin{easylist}

& \textbf{Breadth-first search:} Utilizes a queue
	&& Time and space complexity: Bounded by the number of nodes the same distance away from the root as the goal node
	&& Complete if branching factor is finite

& \textbf{Uniform-cost search:} Utilizes a priority queue to follow the path with least cost
	&& Time and space complexity: Bounded by the branching factor to the power of the average action cost
	&& Complete if branching factor is finite and step costs $\leq \epsilon$

& \textbf{Depth-first search:} Utilizes a stack
	&& Time complexity: Bounded by the maximum depth of the tree
	&& Space complexity: Bounded by the length of the maximum-length path, and its remaining unexpanded child nodes for each node on the path
	&& Not complete
	&& More space-efficient than breadth-first search

& \textbf{Iterative deepening (depth-first) search:} Depth-first search with maximum limited depth, which increases until the goal is found
	&& Complete if branching factor is finite
	&& Time complexity: Bounded by the number of nodes the same distance away from the root as the goal node
	&& Space complexity: Bounded by the length of the goal path, and its remaining unexpanded child nodes for each node on the path

\end{easylist}
\subsection{Informed Algorithms}
	\label{subsec:informed-algorithms}
\begin{easylist}

& \textbf{Admissible heuristic:} Underestimation of the true cost from a given node to the goal
	&& Notation for a function evaluating a heuristic: $h(n)$
	&& Will provide improved time and space complexity over the worst time
	&& To create an admissible heuristic, relax a rule/constraint and find the shortest solution given
	&& \textbf{Manhattan distance:} Number of unrestricted 4-directional moves from the goal state
	&& \textbf{Dominance:} Characteristic of an admissible heuristic which has a result greater than or equal to another admissible heuristic for all $n$
		&& Given admissible heuristics $h_a(n)$ and $h_b(n)$, $h(n) = \textrm{max}\left(h_a(n), h_b(n)\right)$ is also admissible and dominates $h_a(n)$ and $h_b(n)$

& \textbf{Greedy search:} Best-first search algorithm which chooses subsequent nodes by minimizing direct cost to the goal
	&& Complete: Only if state is checked for repeats
	&& Time and space complexity: At worst, bounded by all nodes

& \textbf{A* search:} Best-first search algorithm which expands only the paths with less cost than the optimal cost to the goal
	&& Evaluation function:

	\end{easylist}
	\begin{align*}
		f(n) & = g(n) + h(n) \\
		\textrm{where }
		& f(n) = \textrm{ estimated total cost of the path through } n \textrm{ to the goal} \\
		& g(n) = \textrm{ cost so far to reach } n \\
		& h(n) = \textrm{ heuristic-estimated cost from } n \textrm{ to the goal}
	\end{align*}
	\begin{easylist}

	&& Complete: Yes (unless there are infinitely many nodes with $f \leq f(G)$)

\end{easylist}
\subsection{Complexity of Algorithms}
	\label{subsec:complexity-of-algorithms}

\begin{figure}[!htb]
	\caption{Complexity of Algorithms}
	\label{fig:algos-complexity}
	\center
	\begin{tabular}{ l | c c p{4cm} }
		& Time & Space & Is the Goal Path Optimal? \\
		\hline
		Breadth-first search & $O(b^{d+1})$ & $O(b^d)$ & Only when $d = 1$ or when paths have no cost \\
		Uniform-cost search & $O(b^{\left\lfloor \frac{C^*}{\epsilon} \right\rfloor})$ & $O(b^{\left\lfloor \frac{C^*}{\epsilon} \right\rfloor})$ & Yes \\
		Depth-first search & $O(b^m)$ & $O(b \cdot m)$ & No \\
		Iterative deepening search & $O(b^d)$ & $O(b \cdot d)$ & Yes \\
		\hline
		Greedy search & $O(b^m)$ & $O(b^m)$ & No \\
		A\textsuperscript{*} search & $O(b^m)$ & $O(b^m)$ & Yes \\
		\hline
		Minimax algorithm & $O(b^m)$ (with $\alpha / \beta$, $O(b^\frac{m}{2})$) & $O(b \cdot m)$ & Yes (with rational adversary) \\
		Backtracking search & & & Yes \\
		Hill climbing search & & & No \\
		Simulated annealing search & & & Yes
	\end{tabular}

	\begin{align*}
		\textrm{where }
		b &= \textrm{ branching factor} \\
		d &= \textrm{ depth of the goal} \\
		C^* &= \textrm{ cost of the optimal solution} \\
		\epsilon &= \textrm{ the minimum cost of an action on the goal path} \\
		m &= \textrm{ maximum depth}
	\end{align*}
\end{figure}

\clearpage

	%
% CMPT 310: Artificial Intelligence - A Course Overview
% Section: Game Playing and Adversarial Search
%
% Author: Jeffrey Leung
%

\section{Game Playing and Adversarial Search}
	\label{sec:game-playing-adversarial-search}
\begin{easylist}

& Game playing: Scenario where a game with discrete states is played with two players, each seeking to maximize their position which minimizes the opponent's position
	&& E.g. Chess, checkers
	&& \textbf{Heuristic/evaluation function:} Estimate of the optimality of a game state
	&& \textbf{Terminal state:} Game state which has no further states
	&& Example diagram of game state evaluations in a tree: See figure~\ref{fig:game-tree}
	
\begin{figure}[!htb]
	\caption{Game Tree (Minimizer first)}
	\label{fig:game-tree}
	\centering

	\begin{forest}
		for tree={
			draw,
			circle,
			minimum size=1cm,
			line width=0,
			align=center
		},
		[0
			[2
				[2]
				[-3]
				[-14]
			]
			[14
				[14]
				[17]
				[20]
			]
			[0
				[-12]
				[-18]
				[0]
			]
		]
	\end{forest}
\end{figure}

& \textbf{Minimax algorithm:} Decision-making rule which minimizes the potential loss for a worst-case scenario, given two entities who are alternatively trying to maximize and minimize a score
	&& Performs the equivalent of a depth-first search
	&& Each alternating level of the decision tree is minimized or maximized from the values in the levels below it
	&& Complete: Yes, as long as the tree is finite

& \textbf{$\alpha / \beta$ pruning:} Keeping the maximum/minimum searched values at a given level so far to eliminate nodes which will not exceed a minimum/maximum
	&& $\alpha$: Lower bound on the potential value of a maximum node
	&& $\beta$: Upper bound on the potential value of a minimum node
	&& \href{http://inst.eecs.berkeley.edu/~cs61b/fa14/ta-materials/apps/ab_tree_practice/}{Resource for practice}
	&& See figure~\ref{fig:game-tree-ab} for an example of pruned nodes (notated by dashed edges)
	
\begin{figure}[!htb]
	\caption{Game Tree with $\alpha / \beta$ Pruning (Minimizer first)}
	\label{fig:game-tree-ab}
	\centering

	\begin{forest}
		for tree={
			draw,
			circle,
			minimum size=1cm,
			line width=0,
			align=center
		},
		[0
			[2
				[2]
				[-3]
				[-14]
			]
			[14
				[14]
				[17, edge={densely dashed}]
				[20, edge={densely dashed}]
			]
			[0
				[-12]
				[-18]
				[0]
			]
		]
	\end{forest}
\end{figure}

& Time, space, and optimal complexity of algorithms: See figure~\ref{fig:algos-complexity}

\end{easylist}
\clearpage

	%
% CMPT 310: Artificial Intelligence - A Course Overview
% Section: Constraint Satisfaction Problems
%
% Author: Jeffrey Leung
%

\section{Constraint Satisfaction Problems}
	\label{sec:constraint-satisfaction-problems}
\begin{easylist}

& \textbf{Constraint satisfaction problem (CSP):} Problem where a set of constraints acts on a set of variables, and the solution is a set of variable values which satisfies all constraints
	&& E.g. Sudoku, assiging resources to tasks, map colorings
	&& Formulation of a CSP:
		&&& Specify the variables
		&&& Specify the domain $D$ of possible values for the variables
		&&& Specify contraints

& \textbf{Backtracking search:} Depth-first search which assigns possible values to variables, and backtracks when a variable has no valid values
	&& Complete: Yes
	&& Runtime: $O(n! D^n)$ (actually $O(k^n)$ where $k$ = number of values, $n$ = number of variables)
		&&& For each of $n$ variables, there are $D$ options
		&&&& Unary constraint: Restriction which only applies to a single variable
	&& Space: $O(m)$ where $m$ is the maximum depth
	&& Improvements:
		&&& \textbf{Forward checking:} Assigning to each variable an early set of possibilities restricted by constraints, independent of other variables
		&&& \textbf{Arc consistency:} Whether for each value of a variable there is a valid value for all other variables sharing a binary constraint
			&&&& Runtime complexity of enforcing arc consistency first: $O(n^2 D^2)$
			&&&& $k$ consistency has runtime $O(n^k D^k)$
		&&& Heuristics:
			&&&& \textbf{Minimum-remaining-values heuristic:} Assigning variables starting from those with the least possible number of values, to maximize early failure in order to minimize the number of changed values
			&&&& \textbf{Maximum-degree heuristic:} Assigning variables starting from those which share the most constraints with other variables, to prune future options faster
				&&&&& Used for tiebreakers or starting
			&&&& \textbf{Least-constraining-value heuristic:} Assigning variable values starting from the values which constrain other variables least, to minimize value failure

& \textbf{Local search:} Heuristic which uses iterative improvement on an arbitrary sub-optimal solution
	&& Choose candidates by the variable(s) which violate constraints, using all domain values
	&& Complete: No
	&& Runtime: Unknown
	&& Space: $O(n)$
	&& Methods to choose the next solution:
		&&& \href{https://towardsdatascience.com/simulated-annealing-for-clustering-problems-part-1-3fa8994a3ebb}{\textbf{Hill-climbing search or gradient descent/ascent:}} Local search algorithm which chooses following candidate which violates the fewest constraints
			&&&& Terminates when it reaches a peak where no neighbor has a higher value
			&&&& \textbf{Simulated annealing:} Hill-climbing search algorithm which allows for moderate suboptimal moves (with decreased frequency over time) to escape local maxima and reach a global maximum
				&&&&& Complete: Yes

& \href{https://towardsdatascience.com/introduction-to-genetic-algorithms-including-example-code-e396e98d8bf3}{\textbf{Genetic algorithm:}} Search algorithm maintains a fixed-size set of solutions, from which the optimal individuals are selected to create the successive set of solutions, evolving to a global optimal state over time

\end{easylist}
\clearpage

	%
% CMPT 310: Artificial Intelligence - A Course Overview
% Section: Logical Inference
%
% Author: Jeffrey Leung
%

\section{Logical Inference}
	\label{sec:logical-inference}
\begin{easylist}

& \textbf{Entailment:} Logical connective which means the result is a consequence of the input
	&& Notation: $\alpha \models \beta$

& \textbf{Satisfiability:} Whether a model exists where the logical sentence is true
	&& Includes constraint satisfaction problems
	&& Searches such as: Backtracking, local
	&& \textbf{DPLL algorithm:} Backtracking algorithm which permutes the variables until all clauses are satisfied
		&&& \textbf{Pure symbol:} Symbol which is either never negated or always negated in each clause it appears
		&&& \textbf{Unit clause:} Clause which only contains a single literal
	&& \textbf{WalkSAT algorithm:} Local non-deterministic algorithm which chooses a random false variable and changes it to true to attempt to find a solution
& \textbf{Deduction/Inference:} Given a knowledge base, whether a logical sentence is true (i.e. $KB \models S$)
	&& $KB \models S$ if and only if $KB \land \lnot S$ is unsatisfiable
	&& \textbf{Knowledge base:} Collection of known true sentences

& \textbf{Conjuctive Normal Form (CNF):} Form of a logical statement which is a conjunction of disjunction(s) of literals
	&& Steps to reduce a statement to CNF:
		&&& Replace equivalencies with a conjunction of implications
		&&& Replace implications with $\lnot A \lor B$
		&&& Use DeMorgan's Law to apply negations to clauses
		&&& Use the Distributive Law to reduce clauses to CNF

& \textbf{Inference by resolution:} Process to reduce a set of CNF statements
	&& Given a pair of clauses with complementary literals, resolve them to combine the clauses

\end{easylist}
\clearpage

	%
% STAT 100: Chance and Data Analysis - A Course Overview
% Section: Probability
%
% Author: Jeffrey Leung
%

\section{Probability}
	\label{sec:probability}
\subsection{Introduction}
	\label{subsec:probability:introduction}
\begin{easylist}

	& \emph{Experiment:} Process which creates a bag of outcomes
	& \emph{Sample space:} Set of all possible outcomes
		&& Denoted by $S$
	& \emph{Event:} Possible outcome which satisfies a given condition(s)
		&& Denoted by $E$

	& \emph{Probability:} Value between 0 and 1 (inclusive) representing the proportion of times an outcome/event is expected
		&& Denoted by $P(event)$
		&& Calculation:
		\begin{math}
			P(A) = \frac{|A|}{|S|}
		\end{math}
		where |X| is the number of outcomes for X
		&& 0 means the event is impossible; 1 means the event is certain
		&& Represented either by a fraction or a decimal value
		
		&& E.g. The probability of getting an odd number when rolling a 6-sided die:
		
		\medskip
		\Deactivate
		\begin{center}
			\begin{tabular}{ r @{ = } l }
				$S$ & Results of rolling a 6-sided die = 1, 2, 3, 4, 5, 6 \\
				$|S|$ & 6 \\
				$E$ & Getting an odd number = 1, 3, 5 \\
				$|E|$ & 3 \\
				$P(E)$ & $\frac{|E|}{|S|}$ = $\frac{3}{6}$ = 0.5
			\end{tabular}
			\Activate
		\end{center}
		
	& %TODO


\end{easylist}
\clearpage
	%
% CMPT 310: Artificial Intelligence - A Course Overview
% Section: Machine Learning
%
% Author: Jeffrey Leung
%

\section{Machine Learning}
	\label{sec:machine-learning}
\subsection{Introduction}
	\label{subsec:ml-introduction}
\begin{easylist}

& \textbf{Machine learning:} Approximation of a function which detects the classification of a problem given attributes of data

& \textbf{Classification:} Category of data
& \textbf{Attribute:} Characteristics of a problem using a variable (boolean, discrete, continuous, etc.)
& Epoch: Round of training data

& Types of machine learning:
	&& Supervised learning
	&& \textbf{Semi-supervised learning:} Machine learning technique which classifies data through processing a combination of pre-classified data and unclassified data
	&& \textbf{Unsupervised learning:} Machine learning technique which classifies data and finds patterns through processing previously unclassified data
		&&& E.g. Clustering
	&& \textbf{Reinforcement learning:} Machine learning technique which takes actions to obtain a score, and repeats to find the correct actions to obtain the optimal score
		&&& Used for robotics, games

\end{easylist}
\subsection{Supervised Learning}
	\label{subsec:supervised-learning}
\begin{easylist}

& \textbf{Supervised learning:} Machine learning technique which classifies data through processing provided pre-classified data

& \textbf{Episodic problem (ML):} Situation in supervised learning where a dataset is built from discrete data points
	&& E.g. Spam checker

& \textbf{Inductive learning:} Approximation of a function on the attributes of a problem which produces a label
	&& Notation: Ideal function $f(features) \rightarrow label$; find $h \approx f$
	&& \textbf{Label:} Resulting classification
	&& Highly similar to \href{http://onlinestatbook.com/2/regression/intro.html}{linear regression}
	&& E.g. Finding a best-fit line to a set of datapoints
	&& Choose the simplest and most generalizable model $h$ (Occam's razor)

& \textbf{Hypothesis space:} Set of models $h$ which are valid
	&& Given $n$ boolean functions, there are $2^n$ values in the dataset and $2^{2^n}$ unique possible assignments of results to the dataset
	&& E.g. Neural networks, logistic regressions, support vector machines, random forests
	
& Dataset training:
	&& {Training data:} Dataset which is applied to a neural network to provide neurons with weights and biases
		&&& The greater the data for training, the better the model created
	&& \textbf{Validation data:} Dataset which is applied to a neural network with known optimal results to test accuracy
	&& \textbf{Testing data:} Dataset which is applied to a neural network after training and validation
		&&& Cannot be the same as training data
		&&& The greater the data for testing, the better the estimate of accuracy
	&& Cross-validation: Splitting (`folding') the original dataset into equal subsets, one for testing, one for validation, and the remaining for training
		&&& Strategy: Cross-validate multiple times with each fold to find the most accurate fold
	&& \textbf{Overfitting:} Being overly specific with training data such that testing data is no longer accurate

\end{easylist}
\subsection{Decision Trees}
	\label{subsec:decision-trees}
\begin{easylist}

& \textbf{Decision tree:} Tree where each node represents a decision with multiple choices and each leaf represents a unique choice
	&& Restricting hypothesis space: Limit on depth
	&& Split paths on most informative features (which result in the lowest resulting entropy)
		&&& \textbf{Entropy:} Uncertainty of information (in this context, calculated as similarity of ratio between choices)
			&&&& Notation: $H$ where $0 \leq H \leq 1$
			&&&& Unit: Bits
			&&&& Calculation: $H(p_1, p_2, \dotsc, p_n) = - \sum_i p_i \log_2 p_i$
			&&&& E.g. $H(0.5, 0.5) = - \frac{1}{2} \log \frac{1}{\frac{1}{2}} = 1$
			&&&& E.g. $H(0, 1) = - 0 \cdot \log \frac{1}{0} - 1 \cdot \log \frac{1}{1} = \lim_{x \rightarrow 0} x \log \frac{1}{x} = 0$
			&&&& E.g. $H(\frac{1}{3}, \frac{2}{3}) = 0.918\dots$
		&&& To find the reduction in uncertainty, calculate:

		$$H(root) - \sum_{i \in \textrm{children}} \frac{\textrm{samples of } i}{\textrm{samples of root}} H(i)$$

		&&& See diagram and patrons graph

\end{easylist}
\subsection{Neural Networks}
	\label{subsec:neural-networks}
\begin{easylist}

& \href{https://www.youtube.com/watch?v=aircAruvnKk&list=PLZHQObOWTQDNU6R1_67000Dx_ZCJB-3pi}{\textbf{Neural network:}} Combination of neuron structures through which data flows and is manipulated, to create a set of output values
	&& \textbf{Bias:} Value which is added as a standalone neuron input (typically $-1$)
	&& \textbf{Weight:} Value by which any neuron input is multipled
	&& \href{http://playground.tensorflow.org}{Neural network simulation}

& \textbf{Activation function:} Simple function applied to the sum of weighted neuron inputs to create a neuron output value
	&& Notation: $g: \mathbb{R} \rightarrow \mathbb{R}$
	&& Types of functions:
		&&& \textbf{Threshold function:} Activation function which has the output value -1 for negative input values, and the output value 1 for positive input values
		\end{easylist}
		\[
			a = \begin{dcases}
				-1 & \textrm{if } in < 0 \\
				1 & \textrm{otherwise}
			\end{dcases}
		\]
		\begin{easylist}
		
		&&& \textbf{Sigmoid/logistic function:} Activation function which changes output value gradually from 0 to 1, and has output value 0.5 at input value 0
		\end{easylist}
		\[
			a = \frac{1}{1+exp_e(-in)}
		\]
		\begin{easylist}
		
		&&& \textbf{Rectifier Linear Unit (ReLU) function:} Activation function which has output value 0 for negative values, and output value equal to the input value for positive values
		\end{easylist}
		\[
			a = max(0, in) = \begin{dcases}
				0 & \textrm{if } in \leq 0 \\
				in & \textrm{otherwise}
			\end{dcases}
		\]
		\begin{easylist}
		
& Process of training a network:
	&& \textbf{Squared error:}
	\end{easylist}
	\begin{align*}
		Err
		&= y - h_w(x) \\[1em]
		\textrm{Squared error } E
		&= \frac{Err^2}{2} \\
		&= \frac{(y - h_w(x))^2}{2} \\[1em]
		\textrm{where }
		x &= \textrm{ input value} \\
		y &= \textrm{ true/ideal output value}
	\end{align*}
	\begin{easylist}
	
		&&& Using \href{https://www.youtube.com/watch?v=IHZwWFHWa-w&list=PLZHQObOWTQDNU6R1_67000Dx_ZCJB-3pi&index=2}{gradient descent} to find the optimal value (one iteration per attribute):
		\end{easylist}
		\begin{align*}
			\frac{\partial E}{\partial W_j}
			&= Err \times \frac{\partial E}{\partial W_j} \\
			&= Err \times \frac{\partial}{\partial W_j} (y - g(\sum_{j=0}^n W_j x_j)) \\
			&= -Err \times g'(in) \times x_j
		\end{align*}
		\begin{easylist}
		&&& Given the error, update the weight by:
		\end{easylist}
		\begin{align*}
			W_j &\leftarrow W_j + (\alpha \times Err \times g'(in) \times x_j) \\[1em]
			\textrm{where }
			\alpha
			&= \textrm{ rate of learning correction}
		\end{align*}
		\begin{easylist}
	&& \href{https://www.youtube.com/watch?v=Ilg3gGewQ5U&list=PLZHQObOWTQDNU6R1_67000Dx_ZCJB-3pi&index=3}{\textbf{Back propagation:}} Process of training a neural network by altering weights and biases to provide the intended results, using the outputs of training data
		&&& Involves \href{https://www.youtube.com/watch?v=tIeHLnjs5U8&list=PLZHQObOWTQDNU6R1_67000Dx_ZCJB-3pi&index=4}{calculus}
	\[
		\delta_{out} = (y - a_{out}) \cdot g'(in_{out})
	\]

& \textbf{Single-layer perceptron:} Neural network consisting of a single neuron
	&& \textbf{Linearly separable:} Property of a multidimensional dataset which can be classified using a single linear function
		&&& A function can be represented by a single-layer perceptron if and only if its data is linearly separable

& \textbf{Multi-layer perceptron:} Neural network consisting of sets of multiple interconnected neurons where every node in each layer depends on at least one node in the layer immediately prior, and does not depend on nodes in other layers
	&& \textbf{Hidden layer:} Layer of neurons which is neither the input layer nor the output layer
	&& Acyclic digraph (i.e. has a topological order)
	&& Multi-layer perceptron with 2 hidden layers (3 total layers of processing) can emulate any possible function
	&& \textbf{Multi-task learning:} Model of multi-layer neural network which has consistent hidden layer processing but has multiple unique outputs
		&& E.g. Image content differentiator

& \href{https://adeshpande3.github.io/A-Beginner%27s-Guide-To-Understanding-Convolutional-Neural-Networks/}{\textbf{Convolutional layer:}} Layer component of a neural network which handles prestructured input by abstracting and processing components of the input, often to fewer inputs
	&& E.g. Image consisting of a 2-d grid of pixels, squares of which are taken and compressed
	&& \textbf{Filter:} Structured subset of input values
	&& \textbf{Pooling:} Combination of multiple inputs (often the depth of a 3D matrix) into a single output (e.g. summation, multiplication, averaging, maximization)
& \textbf{Recurrent network:} Acyclic digraph which is structured as a 2D matrix, with a set of input nodes flowing to a set of output nodes
	&& E.g. Conversational bot

\end{easylist}
\clearpage

	%
% CMPT 310: Artificial Intelligence - A Course Overview
% Section: Flashcard Questions
%
% Author: Jeffrey Leung
%

\section{Flashcard Questions}
	\label{sec:flashcard-questions}
\begin{easylist}

& What characteristics does a task environment consist of?
	&& Performance measure, Environment, Actuators, Sensors (PEAS)
& What does a problem formulation consist of?
	&& Initial state, goal test, successor function, and cost function

& How is uniform-cost search different from breadth-first search?
	&& Uniform-cost search uses a priority queue to follow the path of least cost
& Which search algorithm is iterative deepening search derived from?
	&& Depth-first search
& How is iterative deepening search unique?
	&& Maximum depth of depth-first search increases until the goal is found

& What is an admissible heuristic?
	&& Function which underestimates the true cost to the goal
& What is a dominant heuristic?
	&& Admissible heuristic which is greater than or equal to another admissible heuristic
& How is greedy/heuristic search different from A* search?
	&& Heuristic search does not consider the distance already travelled
& What is the equation for A* search?
	\end{easylist}
	\begin{align*}
		f(n) & = g(n) + h(n) \\
		\textrm{where }
		& f(n) = \textrm{ estimated total cost of the path through } n \textrm{ to the goal} \\
		& g(n) = \textrm{ cost so far to reach } n \\
		& h(n) = \textrm{ heuristic-estimated cost from } n \textrm{ to the goal}
	\end{align*}
	\begin{easylist}

& Which of $\alpha / \beta$ is the upper bound on the potential minimum node, and which is the lower bound on the potential maximum node?
	&& $\alpha$: Lower bound on the potential value of a maximum node
	&& $\beta$: Upper bound on the potential value of a minimum node

& When does backtracking search backtrack?
	&& When a variable has no possible valid values, given the assignments of the other variables
& What algorithm is DPLL derived from?
	&& Backtracking search
& How does WalkSAT `walk'?
	&& The value of a random variable from a false clause is flipped

& What is a normalization constant?
	&& A value which every probability is divided by, to reduce a probability function to a total probability of 1
& What is the difference between dependence and conditional dependence?
	&& Dependence: Whether knowing a variable has an effect on the probability of another variable (share a parent node in Bayesian network)
	&& Conditional dependence: Given an evidence variable(s), whether knowing a variable has an effect on the probability of another variable (share a child node in Bayesian network)
& What is the formula for the Chain Rule?
	\end{easylist}
		\begin{align*}
		P(a, b) &= P(a) P(b|a) = P(b) P(a|b) \\
		P(x_1, \dotsc, x_n) &= \prod_{i=1}^n P(x_i | x_1, \dotsc, x_{i-1})
		\end{align*}
	\begin{easylist}

& What is the formula for Bayes' Rule?
	\[
		P(B | A) = \frac{P(A | B) \times P(B)}{P(A)}
	\]
	
& What is rejection sampling?
	&& Sampling each variable multiple times to find approximate probabilities of all variables
& What is Gibbs sampling?
	&& Fixing evidence variables, randomly assigning values to non-evidence variables, sampling a non-evidence variable, and iterating through all non-evidence variables

& What is filtering?
	&& Finding a hidden probability given evidence of all previous probabilities (i.e. finding $x_t$ given $e_{1:t}$)
& What is smoothing?
	&& Finding previous hidden probabilities given evidence of previous and future probabilities (i.e. finding $x_t$ given $e_{1:T}$ where $1 <= t < T$)
& What does the Viterbi algorithm do?
	&& Find the most likely sequence of states ending in xt

& How do you avoid overfitting?
	&& Use relatively less training data

& What is the equation used for entropy?
	&& $H(p_1, p_2, \dotsc, p_n) = - \sum_i p_i \log_2 p_i$
& What is the equation used for reduction in uncertainty?
	&& $H(root) - \sum_{i \in \textrm{children}} \frac{\textrm{samples of } i}{\textrm{samples of root}} H(i)$
& What is the entropy of a 50/50 choice?
	&& 1
& What is the entropy of a 0/100 choice?
	&& 0
& Do you make a choice at a decision tree to increase or decrease entropy?
	&& Decrease entropy (i.e. maximize reduction of entropy)

& What is the equation for a threshold function?
	\end{easylist}
	\[
		a = \begin{dcases}
			-1 & \textrm{if } in < 0 \\
			1 & \textrm{otherwise}
		\end{dcases}
	\]
	\begin{easylist}

& What is the equation for a ReLU function?
	\end{easylist}
	\[
		a = max(0, in)
	\]
	\begin{easylist}

\end{easylist}
\clearpage


\end{document}
