%
% CMPT 354: Database Systems I - A Course Overview
% Section: Design
%
% Author: Jeffrey Leung
%

\section{Design}
	\label{sec:design}
\subsection{Structure}
	\label{subsec:design:structure}
\begin{easylist}
	
	& Conceptual database design:
		&& Identify the entities and relationships (see subsection~\ref{sec:entity-relationship-model})
		&& Identify the entity and relationship information
		&& Identify the integrity constraints
		&& Discuss the conceptual model with the client
		
	& Logical database design:
		&& Decide on a data model to use
		&& Choose a DBMS to use
		&& Create a database schema using the conceptual schema
			&&& Avoid redundant information or inability to record information

\end{easylist}
\subsection{Identification of Tables}
	\label{subsec:design:identification-of-tables}
\begin{easylist}

	& \emph{Key/Superkey:} Set of attributes with values which uniquely identify an entity in an entity set
		&& Mathematical definition: A subset $K$ of a relation $R$ is a superkey of $R$ if, for all pairs of tuples $t_1$ and $t_2$, $t_1 \neq t_2$, then $t_1[K] \neq t_2[K]$
			&&& I.e. $K$, a set of attributes of relation $R$, is a superkey if, for all distinct pairs of tuples, there are no two tuples with the same values for $K$
		&& \emph{Candidate key:} Superkey with no extraneous attributes
			&&& E.g. Class number, teacher + term + meeting times
			&&& \emph{Primary key:} Candidate key chosen to represent a tuple in the relation (rows in the table)
				&&&& Should be chosen from an attribute(s) which never changes (e.g. Social Insurance Number)
				&&&& Can be arbitrarily generated for easier classification
				&&&& In an entity-relationship diagram: Attributes underlined
				
\end{easylist}
\subsection{Constraints}
	\label{subsec:design:constraints}
\begin{easylist}
				
	& \emph{Integrity constraint:} Rule which restricts the data in a database
		&& \emph{Domain constraint:} Integrity constraint on a given data column of the type of data
		&& \emph{Key constraint (integrity constraint):} Integrity constraint which identifies primary keys and candidate keys
		&& \emph{Foreign key constraint:} Integrity constraint which references a primary key from another tables
			&&& \emph{Foreign key:} Attribute(s) which reference a primary key of another entities
				&&&& References the entire primary key
				&&&& Number of attributes and attribute types must be consistent; attribute names may be different

\end{easylist}
\subsection{Normalization}
	\label{subsec:design:normalization}
\begin{easylist}

	& \emph{Normalization:} Minimizing redundancy of related information
		&& Goals:
			&&& Efficient space use
			&&& Efficient processing of data
			&&& Minimal possibility of inconsistent data and therefore data integrity
	
	& \emph{Repetition:} Data which is repeated for no use, and which obscures updating
		&& E.g. A customer can own many accounts; accounts can have many customers. \\
		Account = \{customerID, accNumber, balance, type\} \\
		There will be multiple rows for each account with multiple customers, so updating the information of one customer's account will require updating multiple records.
	
	& OTHER:
		&& \emph{Lossless join:} A join where only and all of the appropriate data is returned %TODO ?
		&& \emph{Lossy join:} A join where extraneous information/records is created %TODO ?
			&& E.g. Decomposing a table into two tables, one of which has no primary key, and joining them together will create multiple records with useless information
			
	& Normalized database design:
		&& Creating the relational database schema:
			&&& Do not use composite attributes or set valued attributes
			&&& Do not assign attributes to relationship sets which are not descriptive
		&& Decompose the tables and make sure they satisfy at least one normal form (see subsection~\ref{subsec:design:normal-forms})

\end{easylist}
\subsection{Normal Forms}
	\label{subsec:design:normal-forms}
\begin{easylist}

	& \emph{First Normal Form:} Placing all data in one table while removing repeating groups
		&& Adds rows for repeated groups
		&& Uses a compound primary key
		&& Creates fixed-length records
		&& Complies with the relational model
		&& Disadvantages:
			&&& Redundant data due to repeated information in groups of rows
			&&& An instance of one attribute of the compound primary key cannot be inserted without creating an instance of the other attribute(s) of the compound primary key
			&&& Deleting the last instance of an attribute in a primary key also deletes... %TODO
			&&& Update... %TODO
		
	& \emph{Second Normal Form:} Removing partial key dependencies from a First Normal Form...
		&& %TODO
		&& Any database in 2NF is also in 1NF
		&& Disadvantages:
			&&& Only considers partial key dependencies; ignores non-key dependencies
			&&& Redundant data
			&&& Insert/delete/update anomalies
	
	& \emph{Third Normal Form:} %TODO
	
	& See \emph{functional dependencies}, subsection %TODO \ref{}
		&& Goals of decomposition:
			&&& Should not result in a lossy join %TODO ref
				&&&& %TODO notes
			&&& \emph{Dependency preservation:} If a set of attributes depends on an attribute, the dependency should be maintained in one table
			&&& Minimal redundancy
			
	& \emph{Boyce-Codd Normal Form (BCNF):} Obtained by simplifying functional dependencies from Third Normal Form
		&& Ignores multi-valued dependencies
		&& The only functional dependencies are those where the key of the table determines attributes (excluding trivial dependencies)
		&& Database is in BCNF if each table is in BCNF
		&& Strict definition: A relational schema $R$ is in BCNF with respect to a set of dependencies $F$ if, for all... %TODO notes
	
	
\end{easylist}
\subsection{Views}
	\label{subsec:relational-model:views}
\begin{easylist}

	& \emph{View:} External schema which displays a particular set of data
		&& Convenient for users to access data without referring to multiple tables
		&& Ensures that users can only access specific data
		&& Masks changes in the conceptual schema
		&& Updates may cause problems when derived from multiple tables, and does not include the primary keys of all tables, so they are generally only allowed on views derived from a single table	

\end{easylist}
\clearpage