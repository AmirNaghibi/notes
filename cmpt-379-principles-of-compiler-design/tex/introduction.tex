%
% CMPT 379: Principles of Compiler Design - A Course Overview
% Section: Introduction
%
% Author: Jeffrey Leung
%

\section{Introduction}
	\label{sec:introduction}
\begin{easylist}

& \textbf{Compiler:} Program which translates a source program from user intention in text form into target in machine code
	&& \textbf{Interpreter:} Program which dynamically executes code

& \textbf{Bootstrapping:} Creating a compiler for a language by building a compiler for a simple subset of the language, then using the subset for the rest of the language definition
	&& Can be done with a different language

& \textbf{Intermediate representation:} Machine code representation of the program which may be modified and optimized before outputting

& Challenges:
	&& Differing architectures, memory hierarchies, memory sizes
	&& Instruction and algorithm parallelism
	&& Branch prediction

& Stages:
	&& \textbf{Analysis (front-end):} Stage of compiling code which reviews and understands the code in lexical, syntax/parsing, and semantic/type-checking contexts
	&& \textbf{Synthesis (back-end):} Stage of compiling code which generates and optimizes code

\end{easylist}
\clearpage
