%
% CMPT 379: Principles of Compiler Design - Syntax Analysis Worksheet
%
% Author: Jeffrey Leung
%

\documentclass[10pt, oneside, letterpaper]{article}

\usepackage{amsmath}
\usepackage[ampersand]{easylist}
	\ListProperties(
		Progressive*=5ex,
		Space=5pt,
		Space*=5pt,
		Style1*=\textbullet\ \ ,
		Style2*=\begin{normalfont}\begin{bfseries}\textendash\end{bfseries}\end{normalfont} \ \ ,
		Style3*=\textasteriskcentered\ \ ,
		Style4*=\begin{normalfont}\begin{bfseries}\textperiodcentered\end{bfseries}\end{normalfont}\ \ ,
		Style5*=\textbullet\ \ ,
		Style6*=\begin{normalfont}\begin{bfseries}\textendash\end{bfseries}\end{normalfont}\ \ ,
		Style7*=\textasteriskcentered\ \ ,
		Style8*=\begin{normalfont}\begin{bfseries}\textperiodcentered\end{bfseries}\end{normalfont}\ \ ,
		Hide1=1,
		Hide2=2,
		Hide3=3,
		Hide4=4,
		Hide5=5,
		Hide6=6,
		Hide7=7,
		Hide8=8 )
\usepackage{forest}
\usepackage{geometry}
	\geometry{margin=1.2in}
\usepackage{graphicx}
	\graphicspath{ {img/} }
\usepackage[colorlinks=true, linkcolor=blue]{hyperref}
\usepackage{listings}
\usepackage{lmodern} % Allows the use of symbols in font size 10; http://ctan.org/pkg/lm
\usepackage{multirow}
\usepackage{textcomp} % Allows the use of \textbullet with the font
\usepackage{verbatim}

\renewcommand{\arraystretch}{1.2}
\renewcommand{\familydefault}{\sfdefault}

\title{CMPT 379: Principles of Compiler Design \\\medskip \Large Syntax Analysis Worksheet (Answers)}
\author{}
\date{}

\begin{document}

	\maketitle

	\begin{enumerate}
	
		\item Given language $\bigg\{ wcw^R | w \in (a|b)* \bigg\}$ ($w^R$ is $w$ in reverse), write a CFG which produces this language.
		\vspace{2cm}
		
		\item Given language $\bigg\{ a^n b^m c^m d^n | n \geq 1, m \geq 1 \bigg\}$, write a CFG which produces this language.
		\vspace{2cm}
		
		\item Does bottom-up parsing create a rightmost or leftmost derivation?
		\vspace{2cm}
		
		\item Where is a handle located?
		\vspace{2cm}
		
		\clearpage
		
		\item Given the production rules in table~\ref{tab:q5-prod-rules}, write out the parse tree for the string $id + id * id$.

\begin{figure}[!htb]
	\caption{Question 5: Production Rules}
	\label{tab:q5-prod-rules}
	\begin{center}
		\begin{tabular}{ l }
			$E \rightarrow E + E$ \\
			$E \rightarrow E * E$ \\
			$E \rightarrow id$
		\end{tabular}
	\end{center}
\end{figure}
		
		\clearpage
		
		\item Given the production rules in table~\ref{tab:q6-prod-rules}, write out the parse tree for the string $int * int + int$.

\begin{figure}[!htb]
	\caption{Question 6: Production Rules}
	\label{tab:q6-prod-rules}
	\begin{center}
		\begin{tabular}{ l }
			$E \rightarrow T + E$ \\
			$E \rightarrow T$ \\
			$T \rightarrow int$ \\
			$T \rightarrow int * T$ \\
			$T \rightarrow (E)$
		\end{tabular}
	\end{center}
\end{figure}
		
		\clearpage
		
		\item Given the production rules and action/goto table in figure~\ref{tab:q7-productions-table}, write the trace of the LR(0) parse.

\begin{figure}[!htb]
	\caption{Question 7: Production Rules and Action/Goto Table}
	\label{tab:q7-productions-table}
	\begin{center}
		\begin{tabular}{ r | l }
			& Production Rules \\
			\hline
			1 & $T \rightarrow F$ \\ 
			2 & $T \rightarrow T*F$ \\
			3 & $F \rightarrow id$ \\
			4 & $F \rightarrow (T)$
		\end{tabular}
		
		\begin{tabular}{ r | c | c | c | c | c || c | c }
			\multirow{2}{*}{\textbf{States}} & \multicolumn{5}{ c || }{\textbf{Actions}} & \multicolumn{2}{c}{\textbf{Gotos}} \\
			& * & ( & ) & id & \$ & T & F \\
			\hline
			0 & & S5 & & S8 & 2 & 1 \\
			1 & R1 & R1 & R1 & R1 & R1 & \\
			2 & S3 & & & & ACC & \\
			3 & & S5 & & S8 & & 4 \\
			4 & R2 & R2 & R2 & R2 & R2 & \\
			5 & & S5 & & S8 & & 6 & 1 \\
			6 & S3 & & S7 & & & \\
			7 & R4 & R4 & R4 & R4 & R4 & \\
			8 & R3 & R3 & R3 & R3 & R3 &
		\end{tabular}
	\end{center}
\end{figure}

		\clearpage

		\item Given the set of production rules and dotted rule in figure~\ref{tab:q8-closure}, find the configuration set resulting from the closure of the dotted rule.

\begin{figure}[!htb]
	\caption{Question 8: Closure}
	\label{tab:q8-closure}
	\begin{center}
		\begin{tabular}{ r | l }
			Production Rules
			& $T \rightarrow F$ \\
			& $T \rightarrow T*F$ \\
			& $F \rightarrow id$ \\
			& $F \rightarrow (T)$ \\
			\hline
			Dotted Rule
			& $T \rightarrow T * \bullet F$ \\
		\end{tabular}
	\end{center}
\end{figure}

		\clearpage

		\item Given a set of production rules, a configuration set, and a successive character in figure~\ref{tab:q9-successor}, apply the successor function to create a new configuration set.

\begin{figure}[!htb]
	\caption{Question 9: Successor}
	\label{tab:q9-successor}
	\begin{center}
		\begin{tabular}{ r | l }
			Production Rules
			& $S' \rightarrow T$ \\
			& $T \rightarrow F$ \\
			& $T \rightarrow T*F$ \\
			& $F \rightarrow id$ \\
			& $F \rightarrow (T)$ \\
			\hline
			Configuration set $I$
			& $S'\rightarrow \bullet T$ \\
			& $T \rightarrow \bullet F$ \\
			& $T \rightarrow \bullet T * F$ \\
			& $F \rightarrow \bullet id$ \\
			& $F \rightarrow \bullet (T)$ \\
		\end{tabular}
	\end{center}
\end{figure}

		\clearpage
	
		\item Given a set of production rules in figure~\ref{tab:q10-first-follow}, find the First and Follow sets of the non-terminal symbols.

\begin{figure}[!htb]
	\caption{Question 10: First/Follow}
	\label{tab:q10-first-follow}
	\begin{center}
		\begin{tabular}{ r | l }
			Production Rules:
			& $A \rightarrow Bc | d$ \\
			& $B \rightarrow e$ \\
		\end{tabular}
	\end{center}
\end{figure}

		\clearpage
	
		\item Given the set of production rules in figure~\ref{tab:q11-first-follow}, find the First and Follow sets of the non-terminal symbols.

\begin{figure}[!htb]
	\caption{Question 11: First/Follow}
	\label{tab:q11-first-follow}
	\begin{center}
		\begin{tabular}{ r | l }
			Production Rules:
			& $A \rightarrow Bc$ \\
			& $B \rightarrow d | BC | Be $ \\
			& $C \rightarrow f$ \\
		\end{tabular}
	\end{center}
\end{figure}

		\clearpage
	
		\item Given a set of production rules in figure~\ref{tab:q12-first-follow}, find the First and Follow sets of the non-terminal symbols.
		
\begin{figure}[!htb]
	\caption{Question 12: First/Follow}
	\label{tab:q12-first-follow}
	\begin{center}
		\begin{tabular}{ r | l }
			Production Rules
			& $S \rightarrow AB$ \\
			& $A \rightarrow c | \epsilon$ \\
			& $B \rightarrow cbB | a$ \\
		\end{tabular}
	\end{center}
\end{figure}

		\clearpage
	
		\item Given a set of production rules in figure~\ref{tab:q13-first-follow}, find the First and Follow sets of the non-terminal symbols.

\begin{figure}[!htb]
	\caption{Question 13: First/Follow}
	\label{tab:q13-first-follow}
	\begin{center}
		\begin{tabular}{ r | l }
			Production Rules
			& $S \rightarrow cAa$ \\
			& $A \rightarrow cB | B$ \\
			& $B \rightarrow bcB | \epsilon$ \\
		\end{tabular}
	\end{center}
\end{figure}

		\clearpage
		
		\item Given left-recursive grammar $A \rightarrow A * B | B; B \rightarrow a$, create an equivalent grammar which is no longer left-recursive.

		\clearpage
		
		\item Given grammar $A \rightarrow Bc | D; B \rightarrow ef; D \rightarrow e$, left-factor this grammar.

		\clearpage
		
		\item Given the set of production rules in figure~\ref{tab:q16-production-rules}, remove left-recursion from the grammar.

\begin{figure}[!htb]
	\caption{Question 16: Production Rules}
	\label{tab:q16-production-rules}
	\begin{center}
		\begin{tabular}{ r | l }
			Production Rules
			& $S \rightarrow Aa | b$ \\
			& $A \rightarrow Ac | Sd | \epsilon$ \\
		\end{tabular}
	\end{center}
\end{figure}

		\clearpage
		
		\item Given the set of production rules in table~\ref{tab:q-ll1-conflict-resolution}) which form an $LL(1)$ grammar, find the following symbols upon which the production rule $Y \rightarrow \epsilon$ is the valid option.

\begin{figure}[!htb]
	\caption{Question 17: LL(1) Conflict Resolution}
	\label{tab:q-ll1-conflict-resolution}
	\begin{center}
		\begin{tabular}{ r l }
			Production Rules
			& $E \rightarrow TX$ \\
			& $X \rightarrow \epsilon$ \\
			& $X \rightarrow +E$ \\
			& $T \rightarrow (E)$ \\
			& $T \rightarrow id Y$ \\
			& $Y \rightarrow *T$ \\
			& $Y \rightarrow \epsilon $
		\end{tabular}
	\end{center}
\end{figure}
		
	\end{enumerate}
		
	\vspace{2cm}

\end{document}
