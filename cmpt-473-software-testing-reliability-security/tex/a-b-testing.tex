%
% CMPT 473: Software Quality Assurance - A Course Overview
% Section: A/B Testing
%
% Author: Jeffrey Leung
%

\section{A/B Testing}
	\label{sec:a-b-testing}
\begin{easylist}

& \textbf{A/B testing:} Hypothesis testing which provides different services to randomly selected individuals
	&& Requires a hypothesis and population to test

& Used to evaluate:
	&& Usability improvements
	&& Perforamnce improvements
	&& Promotion effectiveness
	&& Gradual rollouts

& Possible issues:
	&& Uncontrolled influencing factors
	&& Populations may not be representative
	&& False positives/negatives
		&&& \textbf{p-hacking:} Altering results by executing many tests to compound the effect of false positives/negatives, and choosing exactly when to stop based on the results
		&&& \textbf{Regression to the mean:} Tendency for results to return to relatively normal levels after an extreme event
			&&&& E.g. Poorly performing students are placed in a program, after which their grades improve

& Ways to mitigate issues:
	&& Calculate significance and test amount beforehand, rather than stopping when significance is reached

\end{easylist}
\subsection{Hypothesis Testing}
	\label{subsec:hypothesis-testing}
\begin{easylist}

& T-Test: Comparison between samples of populations
	&& Modeled as a distribution
	&& Requires the data to have a known variance, independence from other factors

& Sequential testing may have bounding criteria for when to stop early

& \textbf{Multi-armed bandit:} Testing technique which determines the best of multiple options based on evidence so far
	&& Requirements:
		&&& Reward probabilities do not change
		&&& Sampling is singular, instantaneous, and independent
		
	&& \textbf{$\epsilon$-greedy strategy:} Multi-armed bandit technique where the greater the previous sample proportion, the more likely the population is sampled
		&&& Sensitive to variance
		
	&& \textbf{Thompson sampling:} Multi-armed bandit technique where the probability of the best arm is chosen
		
\end{easylist}
\clearpage
