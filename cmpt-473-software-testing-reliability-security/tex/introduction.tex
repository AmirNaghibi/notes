%
% CMPT 473: Software Quality Assurance - A Course Overview
% Section: Introduction
%
% Author: Jeffrey Leung
%

\section{Introduction}
	\label{sec:introduction}
\subsection{Software Quality}
	\label{subsec:introduction:software-quality}
\begin{easylist}

& Quality of development process influences quality of resulting software

& Perspectives/roles of the software:
	&& Priorities of \textit{end users}:
		&&& Fulfill its desired purposes
		&&& Produce reliable results upon consistent input
		&&& Handles bad input
		&&& Easy to use
		&&& Responsive
		&&& Integrates well with other software
	&& Priorities of \textit{operations/deployment}:
		&&& Secure from attacks
		&&& Uses appropriate amount of resources
	&& Priorities of \textit{developers}:
		&&& Easily modifiable
		&&& Comprehensible
		&&& Has measurable quality
		&&& Adaptible to other systems

& \textbf{ISO/IEC 9126:} Functionality, reliability, usability, efficiency, portability, maintainability
	&& \textbf{Reliability:} Characteristic of software which, in the context of software faults, involves avoidance, maintenance of performance during, and re-establishment of performance/data afterwards
	&& \textbf{Usability:} Characteristic of software which is understandable in the context of how it fits the users' needs, easy to learn/operate, and enjoyable
	&& \textbf{Maintainability:} Characteristic of software which makes defects easy to identify, allows changes unlikely to affect other components, and easy to test

& Defect terminology:
	&& \textbf{Defect/fault:} Flaw in static software/code
		&&& \textbf{Latent defect:} Unobserved defect in delivered software which was not exposed by testing
	&& \textbf{Failure:} Observable behaviour which does not match expectations
	&& \textbf{Error/infection:} Not-yet-observed incorrect state

\end{easylist}
\subsection{Quality Measurement}
	\label{subsec:introduction:quality-measurement}
\begin{easylist}

& \textbf{Planning:} Choosing the most important assessment criteria

& Tools:
	&& \textbf{Synthetic tools:} Quality measurement tools/techniques to create better software
		&&& E.g. Design methodologies, coding standards, templates, compilers
	&& \textbf{Analytical tools:} Quality measurement tools/techniques to evaluate software quality
		&&& E.g. Walk-throughs, audits, unit/integration/system testing
	&& \textbf{Manual tools:} Quality measurement tools/techniques which is interactively driven
		&&& E.g. Design methodologies
	&& \textbf{Automated tools:} Quality measurement tools/techniques which requires no interaction
		&&& E.g. Compilers, program generators

& Testing is difficult because of dependencies

& Use polymorphism to create a mock to isolate modules during testing
	&& Inheritance can be used
	&& \textbf{Parametric polymorphism:} Applying superclasses or templates of parameters to allow generics for testing
	&& \textbf{Dependency injection:} Using dependencies by accepting them as arguments upon construction rather than instantiating them directly

& \textbf{Test frame:} Plan for a set of test cases based on partitioned inputs

& Coverage effectiveness:
	&& For statement coverage, having quantitatively more coverage is not necessarily more effective
	&& For mutation testing, test frame size correlates with defect-finding ability

\end{easylist}
\clearpage
