%
% CMPT 473: Software Quality Assurance - A Course Overview
% Section: Mutation Analysis and Testing
%
% Author: Jeffrey Leung
%

\section{Mutation Analysis and Testing}
	\label{sec:mutation-analysis-and-testing}

\subsection{Mutant Fundamentals}
	\label{subsec:mutation-analysis-and-testing:mutant-fundamentals}
\begin{easylist}

& \textbf{Mutant:} Valid program which behaves differently from the original
	&& Involves smallest possible changes
	&& Invalid (and not counted in the mutation score) if:
		&&& Not compilable (still born)
		&&& Killed by most test cases (trivial)
		&&& Equivalent to the original program or to other mutants (redundant; can be undecidable)
	&& A valid mutant must satisfy the reachability, infection, and propagation model:
		&&& \textbf{Reachability:} Ability of the fault to be executed by a test
		&&& \textbf{Infection:} Ability of a fault to cause the program state to differ
		&&& \textbf{Propagation:} Ability of the differing program state to affect the output

& \textbf{Mutation operator:} Systematic change applied to produce a mutant
	&& \textbf{Intraprocedural mutations:} Modifying the internal values or operators of a procedure
		&&& E.g. (Optionally negated) absolute value insertion, arithmetic/relational/conditional operator replacement
	&& \textbf{Interprocedural mutations:} Modifying the inputs of a procedure
		&&& E.g. Parameter values, call target, incoming dependencies

& \textbf{Kill:} Characteristic of a test which produces a different outcome on a mutant than the original program
	&& Formal definition: A test $t$ kills a mutant $m$ if $t$ produces a different outcome on $m$ than the original program
	&& \textbf{Weakly kill:} A mutant test which results in different internal state
		&&& Satisfies reachability and infection, but not propagation
	&& \textbf{Strongly kill:} A mutant test which results in different output
		&&& Satisfies reachability, infection, and propagation

& Difficulties:
	&& Managing and executing a large amount of mutants
	&& Identifying identical mutants

\end{easylist}
\subsection{Fault Seeding}
	\label{subsec:mutation-analysis-and-testing:fault-seeding}
\begin{easylist}

& \textbf{Fault seeding:} Inserting expected faults to be killed
	&& Equation: $$\frac{\textrm{\# of mutants which killed a bug}}{\textrm{\# of mutants}}$$
	&& Issues:
		&&& Faults may not be meaningful
		&&& May forget to remove the faults

\end{easylist}
\subsection{Mutation Testing}
	\label{subsec:mutation-analysis-and-testing:mutantion-testing}
\begin{easylist}

& \textbf{Mutation analysis:} Ability to find bugs using a mutant
& \textbf{Mutation testing:} Process of creating a test suite which covers a representative set of mutants
	&& Given an unkilled mutant, improve the test suite by adding a test which kills it
	&& \textbf{Representative set:} Set of mutants which covers all possible faults

& \textbf{Mutation score:} Quantitative score of mutation analysis effectiveness
	&& Invalid mutants are not counted
	&& Equation: $$\frac
			{\textrm{\# of non-duplicated mutants which kill a bug}}
			{\textrm{\# of non-equivalent, non-duplicated mutants}}
			$$

& Manage scalability by:
	&& Filter based on coverage
	&& Short circuit tests
	&& Testing multiple mutants simultaneously

& Test coverage:
	&& \textbf{Weak mutation coverage}: For each mutant, the test suite contains a test which weakly kills the mutant
	&& \textbf{Strong mutation coverage}: For each mutant, the test suite contains a test which strongly kills the mutant

\end{easylist}
\clearpage
