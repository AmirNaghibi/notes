%
% IAT 210: Introduction to Game Studies - A Course Overview
%
% Author: Jeffrey Leung
%

\documentclass[10pt, oneside, letterpaper, titlepage]{article}

\usepackage[ampersand]{easylist}
	\ListProperties(
		Progressive*=5ex,
		Space=5pt,
		Space*=5pt,
		Style1*=\textbullet\ \ ,
		Style2*=\begin{normalfont}\begin{bfseries}\textendash\end{bfseries}\end{normalfont} \ \ ,
		Style3*=\textasteriskcentered\ \ ,
		Style4*=\textperiodcentered\ \ ,
		Style5*=\textbullet\ \ ,
		Style6*=\begin{normalfont}\begin{bfseries}\textendash\end{bfseries}\end{normalfont}\ \ ,
		Style7*=\textasteriskcentered\ \ ,
		Style8*=\textperiodcentered\ \ ,
		Hide1=1,
		Hide2=2,
		Hide3=3,
		Hide4=4,
		Hide5=5,
		Hide6=6,
		Hide7=7,
		Hide8=8 )
\usepackage{geometry}
	\geometry{margin=1.2in}
\usepackage[colorlinks=true, linkcolor=blue]{hyperref}
\usepackage{verbatim}

\renewcommand{\arraystretch}{1.2}
\renewcommand{\familydefault}{\sfdefault}

\title{IAT 210: Introduction to Game Studies \\\medskip \Large A Course Overview}
\author{Jeffrey Leung \\ Simon Fraser University}
\date{Summer 2015}

\begin{document}

	\maketitle
	\tableofcontents
	\clearpage
	
	%
% IAT 210: Introduction to Game Studies - A Course Overview
% History of Play and Games
%
% Author: Jeffrey Leung
%

\section{History of Play and Games}
	\label{sec:history-of-play-and-games}
\subsection{General Background}
	\label{subsec:history-of-play-and-games:general-background}
\begin{easylist}

	& \emph{Game:} Series of interesting, meaningful choices in pursuit of a clear and compelling goal
		&& Definitions from various experts:
			&&& A game is the voluntary effort to overcome unnecessary obstacles. \emph{-- Bernard Suits}
			&&& A game is an activity among two or more decision makers seekking to achieve their objectives in a limiting context. \emph{-- Clark Abt}
			&&& A game is an art in which players make decisions in order to manage resources in pursuit of a goal. \emph{-- Greg Costikyan}
			&&& A game is an exercise of voluntary control systems in which there is a contest of power, confined by rules, to produce a disequilibrial outcome. \emph{-- Brian Sutton-Smith}
			&&& A formal game has a two-fold structure based on ends (a winning condition) and means (rules by which you win). \emph{--David Parlett}
	
	& \emph{Demystification:} Games' origins are often rooted in traditions
		&& E.g. Masks were sacred objects, checkers was for divination
		
	& Play, fun, and games:
		&& \emph{Homo Ludens} by Johan Huizinga discusses the importance of play in society
		&& Play is a necessary condition for civilization
			&&& E.g. Release of tensions, diplomacy
		&& Characteristics of play:
			&&& Free activity
			&&& Fun; not serious; no material interest
			&&& Bounded; separate from the real world
			&&& Immersive

	& \emph{Gary Gygax:}
		&& Avid wargamer
		&& Responsible for developing:
			&&& The D20 combat system
			&&& The ruleset for \emph{Chainmail}, the origin of \emph{Dungeons and Dragons}
			&&& \emph{Dungeons and Dragons:} Tabletop roleplaying game
				&&&& Players assume the roles and attributes of fantasy characters who embark on magical adventures

\end{easylist}
\subsection{Digital Games}
	\label{subsec:history-of-play-and-games:digital-games}
\begin{easylist}
				
	& Arcade games:
		&& 1890s: Simply streetside amusements
		&& 1940s: Electromechanical projection machines became popular after WWII
		&& 1950s: Pinball games developed
		&& 1970-80s: Arcades became prevalent in malls
		
	& \emph{Nolan Bushnell:}
		&& Creator of Chuck E Cheese's
		&& Invented the Atari game system
		&& Engineer to entrepreneur
		&& Employed Steve Wozniak and Steve Jobs, who:
			&&& Originally worked with Atari on the arcade game \emph{Breakout}
			&&& Created the first Apple computer with Atari parts
			&&& Asked Bushnell to invest \$50,000 in their startup, who declined
		&& Inspired by B.F. Skinner:
			&&& Coined operant conditioning
				&&&& Giving consistent positive/negative feedback to someone to condition them to act that way more/less
			&&& \emph{Skinner box:} Psychological tool which dispensed rewards when a test subject activated a lever
				&&&& Led to one-armed bandits (gambling machines which only require pulling a lever)
	
	& Personal computers created possibilities for game developers
	& \emph{Johann von Neumann:} Contributor to the digital computer
		&& Played games such as kriegspiel (miniature war re-enactments)
		&& Began a more formal analysis of parlour games, leading to the study of game theory itself with a wide range of applications
		&& Defined games as conflicts between players:
			&&& Players are assumed to be 'rational' (i.e. will pursue maximum utility)
			&&& \emph{Minimax theorem:} There is always a rational solution to games where players are in conflict
			&&& Solutions can result in a positive/negative/zero sum game

\end{easylist}
\clearpage
	%
% IAT 210: Introduction to Game Studies - A Course Overview
% Properties of Games
%
% Author: Jeffrey Leung
%

\section{Properties of Games}
	\label{sec:properties-of-games}
\subsection{Games Today}
	\label{subsec:properties-of-games:games-today}
\begin{easylist}

	& Consist of:
		&& Hardware and software components (generally)
			&&& Use a device/platform
		&& Immediate interaction
		&& Data-driven experiences
		&& Increasing complexities
	& Are designed for a specific player personae (see subsection~\ref{subsec:the-audience:player-personae})
	& Are classified and analyzed by genre (see subsection~\ref{subsec:types-of-games:genres})
	& Are created by designers with motives (see section~\ref{sec:reasons-for-game-design})
	& Can have a narrative aspect (see section~\ref{sec:narrative})

\end{easylist}
\subsection{Attributes of Games}
	\label{subsec:properties-of-games:attributes-of-games}
\begin{easylist}

	& A game has:
		&& A winning condition
		&& Mechanics defined by objects, attributes, and actions
		&& Rules to set bounds and show how mechanics lead to the winning condition

	& Game axes:
		&& Rules/play
		&& Casual/hardcore
		&& Challenge/flow
		&& Ludology/storytelling
		&& Uniqueness
			
	& \emph{Fourteen Forms of Fun} by Pierre-Alexandre Garneau
		&& Beauty, immersion, problem solving, competition, social interaction, comedy, \\ thrills/danger, phyiscal activity, love, creation, power discovery, advancement, using an ability
		
	& \emph{Agency:} Autonomy of a player within the constraints of the game
	& \emph{Resources:} Properties which the player may use and control to complete tasks
		&& Provide a goal or incentive
		&& E.g. Money, health/energy, time, space/territory, inventory
		
	& \emph{Fog of war:} Player(s) have limited knowledge about the state of the game
		&& E.g. In StarCraft, any area without the player's troops is darkened and does not show enemy troops
		
\end{easylist}
\subsection{Game Boundaries and Gamespaces}
	\label{subsec:properties-of-games:game-boundaries-and-gamespaces}
\begin{easylist}

	& Game boundaries / gamespace:
		&& Created when players decide that the game begins
		&& \emph{Magic circle:} Finite space with infinite possibilities which sets the mood, tone, and boundaries
			&&& Marks boundary of the game space concretely or abstractly
			&&& Defines an enclosed, separate, temporary space
			&&& Contains the role-playing and experimentation
			&&& Breaking the magic circle ends the game
			&&& E.g. Loading a game into a system and seeing logos, music, and menus
			&&& Examples in other contexts:
				&&&& Lights dimming and trailers appearing in a movie theatre to prepare for the showing of the movie
				&&&& Opening ceremonies to gather attention and a feeling of having begun a special period
		&& \emph{Frame:} Boundary which signifies that a game is being played
			&&& Separate from the real world
			&&& Psychological; can have physical components
		&& \emph{Lusory attitude:} Being in the `mood' to play a game
		&& \emph{Discrete gamespace:} Play where the moves are restricted to a certain set
		&& \emph{Continuous gamespace:} Play where the moves are free and generally unlimited
		&& \emph{Bound:} Boundary or restriction of the  physical gamespace
		&& \emph{Subspace:}	Separate area of play which is governed by slightly different rules and conventions
			&&& E.g. Penalty shots in hockey
		
\end{easylist}
\subsection{Game Mechanics}
	\label{subsec:properties-of-games:game-mechanics}
\begin{easylist}

	& \emph{Game mechanic:} Factor which gives a reward when a certain action happens
		&& Creates contraints which are concrete, interesting, repeatable, and fun
		&& Determines the methods of playing/winning
		&& Creates a difficulty scale
		&& Incorporates the game's flavour
		&& Logical or illogical as required
		&& May be an object, attribute, or action
	& \emph{Object}: Character, prop, token, or other 'physical' component of a game
		&& Noun of mechanical analysis
		&& One or more attributes
			
	& \emph{Attribute:} Property of an object
		&& Detail of mechanical analysis
		&& Static or dynamic; inherent or customizable
			&&& Design needs to know how and when states change
		&& Degree of transparency of attributes affects gameplay
		&& Increase through levelling up/gaining experience
		&& E.g. In \emph{World of Warcraft}, the basic building blocks of a character are:
			&&& Primary: Strength, agility, intellect, stamina etc.
			&&& Secondary: Damage absorption, armour class, spell power, etc.
		&& E.g. In \emph{Need for Speed}, the specifications of your car are:
			&&& Primary: Speed, acceleration, control, strength, durability, etc.
			&&& Secondary: Colour, manufacturer, make, etc.
			
	& \emph{Action:} Verb of mechanical analysis
		&& \emph{Operative action:} Basic, simple action which the player can take
		&& \emph{Resultant action:} Action taken in pursuit of a broader goal
		&& The greater the `operative-resultant ratio', the more unique ways to win

\end{easylist}
\subsection{Rules}
	\label{subsec:properties-of-games:rules}
\begin{easylist}

	& \emph{Rule:} Statement which defines the relationship between players, goals, mechanics, and the game space
		&& Binding; limits player action
		&& Explicit and unambiguous
		&& Shared by all players
		&& Fixed
	& \emph{Operational rule:} Rule which describes how the game is played in general
	& \emph{Foundational rule:} Rule which describes the underlying formal structure of the game
		&& Often tracked in physical games with items such as dice, chips, and game boards
	& \emph{Behavioural rule:} Rule which specifies a social contract between players
		&& Preserving the magic circle
		&& Sportsmanship
	& \emph{Written rule:} Precise documentation which specifies the operative actions
	& \emph{Tournament rule:} Rule which specifies how the game should be played competitively
		&& Eliminates ambiguity
		&& Creates a level playing field
	& \emph{Advisory rule:} Tip or strategy to optimize gameplay
	& \emph{House rule:} Rule created by particular groups of players to suit circumstances or preferences
	
	& Digital games allow for faster, more enjoyable, and more complex gameplay due to being able to enforce rules objectively and almost instantaneously

	& See \emph{Designing Rules}, subsection~\ref{subsec:designing-games:designing-rules}

\end{easylist}
\clearpage
	%
% IAT 210: Introduction to Game Studies - A Course Overview
% Types of Games
%
% Author: Jeffrey Leung
%

\section{Types of Games}
	\label{sec:types-of-games}
\subsection{Types of Play}
	\label{subsec:types-of-games:types-of-play}
\begin{easylist}

	& \emph{Man, Play, and Games} by Roger Callois
		&& Modes of play:
			&&& \emph{Ludus:} Structured play created with the idea of discipline, training, and boundaries
				&&&& Based on patience, skill, and/or ingenuity
				&&&& Less freeform
			&&& \emph{Paidia:} Freeform play
				&&&& Based on spontaneity and/or luck
				&&&& Prioritizes fantasy over structures and limits
		&& Categories of play:
			&&& \emph{Agon:} Competitive play (e.g. chess, sports, contests)
			&&& \emph{Alea:} Chance-based play (i.e. probability and luck)
			&&& \emph{Mimicry:} Role-playing/make-believe (e.g. theatre)
			&&& \emph{Ilinx:} Physical sensation of vertigo (e.g. rollercoasters; children spinning until they fall down)

\end{easylist}
\subsection{Social Games}
	\label{subsec:types-of-games:social-games}
\begin{easylist}

	& Began with brower-based games played and social networks to link such games
		&& Increased in popularity due to companies such as Zynga, Playfish, etc.
	& Involve gameplay requiring cooperation and interaction
		&& \emph{Asynchronous gameplay:} Players do not have to play the game at the same time to have an interactive experience
	
\end{easylist}
\subsection{Mobile Games}
	\label{subsec:types-of-games:mobile-games}
\begin{easylist}

	& Defined by their device(s), which:
		&& Are low-powered
		&& Have a small screen
		&& Are different from portable game devices
	& Allow for interrupted gameplay
	& Focus on mechanics over graphics or story
	
	& History:
		&& Periodic and bumpy
		&& Global market:
			&&& 2005: \$2.6 billion
			&&& 2014: \$10.3 billion
			
		&& Late 1990s:
			&&& Began to emerge
			&&& Constrained by hardware and software limitations
			&&& Defined a new mode of play
			&&& E.g. Tetris, Snake
			
		&& Early 2000s:
			&&& Nokia releaseed the N-Gage phone which:
				&&&& Awkwardly changed the form factor of the phone to suit gaming
				&&&& Attempted to bring big ideas to a small screen
				&&&& Only sold 3 million units
				&&&& Was rebranded as a service rather than hardware
				&&&& Ended sales in 2010
				&&&& Successful games: Pathway to Glory, Rifts
			&&& Apple developed the iTunes system which:
				&&&& Provided:
					&&&&& Automatically downloadable and extendable content
					&&&&& `Frictionless' publishing by third parties
				&&&& Was a surprise hit
				&&&& Established microtransactions as a new consumer behaviour
					&&&&& Enabled proliferation of free/lite games
					
	& Business models:
		&& Ad-supported and/or free
		&& \emph{Freemium:} Game which is essentially free, but requires payment to make any significant progress
		&& Paid
		&& Pay-to-win
		
	& \emph{Kim Kardashian: Hollywood}
		&& Developed by Glu
		&& Simple; linear
		&& Uses a mobile RPG engine reskinned with a Kardashian theme
		&& Freemium:
			&&& Free to download and play
			&&& Not fun and quite difficult by default
			&&& Progress speeds up and fun increases when players pay
		&& Great alignment with Kardashian's fanbase
			&&& Smartpone addicts
			&&& Heavy use of social media
		&& Strong brand alignment which links with other Kardashian media properties
		&& Business statistics (over 3 months):
			&&& 22 million downloads
			&&& \$43 million earned
			&&& 5.7 billion minutes played
	
\end{easylist}
\subsection{Indie Games}
	\label{subsec:types-of-games:indie-games}
\begin{easylist}

	& \emph{Indie game:} Game created by a small-scale developer (individual or team) with little financial support
	
	& Developed from the shareware movement of the 1990s
		&& Technology and tools became more broadly accessible
		&& Increasingly bizzare and uniquegames developed
	& For the developer(s), creating an indie game is:
		&& Personally fulfilling
		&& Not necessarily financially rewarding
		&& A great way to put ideas into practice with little downside
		&& Free of constraints imposed by big budgets, big publishers, etc. to allow for a wide range of experiementation in gameplay/styles
	& A business first and foremost
	& Industry is driven by summits, festivals, game jams
	& Development of open distribution platforms means that indie games will continue to have a future
	& E.g. Minecraft, Super Meat Boy, Mountain
	& For information on the business of indie games, see subsection~\ref{subsec:business:business-models-of-game-types}
	
\end{easylist}
\subsection{Serious Games}
	\label{subsec:types-of-games:serious-games}
\begin{easylist}

	& \emph{Serious game:} Game which attempts to convey a social/moral message to a specific audience
	& E.g. SimEarth, Points of Entry, World Without Oil, September 12\textsuperscript{th}
	& Seek to affect their audience through:
		&& Empowering
		&& Persuading
		&& Educating
		&& Converting
		&& Subverting
		&& Training
	& Do not include:
		&& Education + entertainment games (e.g. typing games)
		&& Education/learning games
		&& Simulations
	& Have a target audience of:
		&& Students
		&& \emph{Barometric citizen:} Someone who potentially may have a great impact
		&& Certain consumers
	& Are models which provide entertaining solutions to real world problems
	
	& Ian Bogost:
		&& Games are models we create
		&& Through designing a game, we impose rules
		&& Rules define roles for players
		&& Player roles lead to understanding or empathy
		&& Understanding is linked to the context of the game and creates meaningful insight
	& E.g. \emph{September 12\textsuperscript{th}} using Bogost's model:
		&& Models a war in the Middle East
		&& Rules consist of having the ability to fire down delayed missiles at civilians and terrorists
		&& Role is the player in the air killing the people
		&& Understanding that killing the 'bad guys' is not so simple
		&& Context is the current war in Iraq
	& \emph{Militainment:} Simulating war in a game to affect a player's thinking
		&& US Military \emph{Doom} mod
			&&& Created 1996
			&&& Used for squad-based training
		&& \emph{America's Army:} First-person shooter designed to recruit youths
			&&& Created by the US Military
			&&& Self-proclaimed propaganda
			&&& Gamified recruiting tool
				&&&& Embodies core values of soldiers
				&&&& Identifies and indoctrinates promising youth while weeding those who would be a bad fit
				&&&& Effective ROI (return on investment)
		&& \emph{Medal of Honor} controversy:
			&&& Players could play as the enemy in the Middle East
			&&& Some were displeased that people could play as the 'enemy'
		&& \emph{This War of Mine:} Survival game where the player has to live as a civilian in a war zone
			&&& Focuses on civilians rather than soldiers
			&&& Created to show life during the siege of Sarajevo
			&&& Survival sub-genre of adventure games
			&&& Using Bogost's model:
				&&&& Models the reality of the civilians behind the soldiers
				&&&& Rules govern possible survival strategies
				&&&& Role is to keep the family alive until the siege ends
				&&&& Understanding is built through choice and moral dilemma
				&&&& Context is a war in a civilian area	
	
\end{easylist}
\subsection{Platforms}
	\label{subsec:types-of-games:platforms}
\begin{easylist}

	& PC:
		&& One of the main platforms
		&& Large role in the market
		&& Interface: Mouse and keyboard
			&&& Allows for  more complex interactions
		
	& Consoles: Dedicated game box which requires a TV to be used
		&& Began with the Atari 2600
		&& Currently: Xbox, PlayStation, Wii
		&& Interface: Controllers of increasing complexity
		
	& Smartphones:
		&& Began with the iPhone (released in 2007)
		&& `Must-have' consumer device; pushed portable gaming devices out of the market
		&& Gameplay defined by mobile device considerations
		&& Interface: Touchscreen controls
			&&& Allows for more intuitive interactions
	
	& Tablets:
		&& Larger screens allow for a richer experience
		&& Popular with the especially older and younger generations
		&& Interface: See \emph{Smartphones}, above

\end{easylist}
\subsection{Genres}
	\label{subsec:types-of-games:genres}
\begin{easylist}

	& \emph{Genre:} Name which encapsulates a type of objects defined by specific and recurring characteristics
		&& Term used to categorize arts and entertainment
	
	& Puzzle:
		&& \emph{Static puzzle:} Puzzle in which the game board does not change significantly
			&&& E.g. Crossword puzzles
		&& \emph{Dynamic puzzle:} Puzzle in which the game board changes significantly during gameplay
			&&& E.g. Tetris
		&& Popular category of casual games
		&& Generally require logic and analysis
		&& E.g. Tetris, Puzzle Pirates, 2048, Angry Birds
		
	& \emph{Simulation:} Representation of one system of behaviour through another abstracted system
		&& May revolve around resource management and development
		&& E.g. The Sims, Flight Simulator
		
	& Strategy:
		&& Also called `god games' due to the player's perspective
		&& Turn-based or RTS (real time strategy)
		&& 4 X's of objectives:
			&&& Explore
			&&& Expand
			&&& Exploit
			&&& Exterminate
		&& Rely on analysis and resource management
		&& Pushed classic war-games out of the market
		&& E.g. Civilization, Alpha Centauri, Total War
	
	& Adventure:
		&& Began as text-based games
		&& Can combine storytelling and puzzles
		&& Less twitch and reaction
		&& E.g. Fable, Blade Runner, Myst
		
	& Shooters:
		&& FPS (first-person shooter), TPS (third-person shooter), platformer
		&& Fast-paced, twitch-based reaction
		&& Simple; easy to understand
		&& Often played online with others
		&& E.g. Halo, Doom, Call of Duty
		
	& Role-playing game (RPG):
		&& Extension of Dungeons and Dragons (see \emph{Gary Gygax}, subsection~\ref{subsec:history-of-play-and-games:general-background})
		&& Players assume the role of a fantasy character and create development through adventures
		&& Utilizes the dice combat system, leveling, classes, abilities
		&& Massively Multiplayer Online Role Playing Games (MMORPGs) greatly expanded the audience and appeal
		&& E.g. Mass Effect, World of Warcraft, EVE Online
		
	& Sports:
		&& Meant to emulate the physical sport
		&& Incredibly successful
		&& Difficult to develop/expand from
		&& E.g. Madden, Fifa, NHL, NBA
	
	& Party:
		&& Casual gameplay
		&& `Living room' experience for multiple players
		&& Can range from mini-games to physical play
		&& Fun revolves around participation and humour
		&& E.g. Dance Dance Revolution, Mario Party, Wario Ware, Guitar Hero

\end{easylist}
\clearpage
	%
% IAT 210: Introduction to Game Studies - A Course Overview
% Narrative
%
% Author: Jeffrey Leung
%

\section{Narrative}
	\label{sec:narrative}
\begin{easylist}

	& \emph{Story:} Experience bounded in time with a beginning, middle, and end (in any order)
	& Examples of narrative frameworks:
		&& 3-act structure
		&& Hero's journey
		&& Non-linear stories
	& Types of story designs:
		&& \emph{Classical design:}
			&&& Active protagonist
			&&& Struggle against external forces
			&&& Continuous time
			&&& Consistent reality
			&&& Closed ending
		&& \emph{Minimalist structure:}
			&&& Passive protagonist
			&&& Struggle against internal forces
			&&& Consistent reality
			&&& Open ending
		&& \emph{Antiplot narrative:}
			&&& Non-linear time
			&&& Inconsistent reality
			&&& Chance and coincidence are prevalent
	& \emph{Run, Lola, Run:} Movie by Tom Twyker from 1998
		&& Combines elements of all 3 story design types
		&& Can be thought of as a narrative database from which many stories could be told
		&& Utilizes common film conventions such as:
			&&& Flashbacks
			&&& Foreshadowing
			&&& Hand-held filming
			&&& Animation
			&&& Supernatural elements
			&&& \emph{Maguffin:} Motivator which a character pursues with little or no narrative explanation/reason
		&& Characteristics of classical design:
			&&& Active protagonist - Lola runs to her father for money
			&&& Struggle against external forces: Time, the hobo with the money, the police
			&&& Continuous time in each part
			&&& Closed ending - death of either, or both surviving
		&& Characteristics of non-linear films:
			&&& Time repeats and events change
			&&& Choice and coincidence from small changes
			&&& ``Hero's Journey'':
				&&&& Coined by Joseph Campbell in the book \emph{The Hero with a Thousand Faces}
				&&&& Made more accessible (adapted to film) by Christopher Vogler
				&&&& Use of recognizable and universal archetypes (e.g. threshold guardians, tricksters, mentors)
		&& Similarities to a video game:
			&&& Having `lives' or starting over with a new/better strategy
			&&& Bounded game space - city of Berlin
			
	& \emph{Narrative gestalt:} How a game and its story are connected and influence each other
		&& How the story emerges from the game based on your actions and choices
		
\end{easylist}
\clearpage
	%
% IAT 210: Introduction to Game Studies - A Course Overview
% Participatory Gaming
%
% Author: Jeffrey Leung
%

\section{Participatory Gaming}
	\label{sec:participatory-gaming}
\subsection{Subculture}
	\label{subsec:participatory-gaming:subculture}
\begin{easylist}

	& \emph{Subculture:} Culture within (usually a rejection of) a mainstream culture
	& Maintains basic mainstream culture attributes but changes attitudes, values, interests, terminology, music, art, etc.
	& Primarily demonstrated by teens and young adults
	& Forming a tight in-group and rejecting those who do not adopt the same values
	& \emph{In-group:} Cohesive social group that members identify with, and members associate with other members
		&& \emph{In-group bias:} In-group members favour in-group members over others
	& \emph{Out-group:} Any person who a member of an in-group does not identify with
		&& \emph{Out-group bias:} In-group members are hostile towards out-group members
	& \emph{Poseur:} Person who adopts subculture identifiers for acceptance into the group, but does not truly recognize the core values of the group
		&& Perceived as inauthentic
				
	& Nerd subculture:
		&& \emph{Nerd:} Person obsessed with obscure or unpopular hobbies
		&& Formed around 'nerdy' hobbies and interests such as video games, anime, comic books, etc.
		&& Looser than other subcultures

		&& History and growth:
				
			&&& Once mocked, now proud and celebrated in popular culture
			&&& Became mainstream due to the internet, and its facilitation of communication through groups dedicated to specific topics:
				&&&& BBS and Usenet:
					&&&&& Exchange of ideas
					&&&&& Small audience, little exposure to new ideas/interests
				&&&& Forums:
					&&&&& Indexed by search engines
					&&&&& Very small subcultures
			&&& Web 2.0 allowed massive social media and content-sharing websites (e.g. Facebook, 4chan, Tumblr, Reddit)
				&&&& No dedicated discussion topic
				&&&& Extremely rapid and diverse exchange of ideas and content to a wide audience
				&&&& Allowed discussion of taboo hobbies and topics
				&&&& Helped expose nerd culture
						
		&& Identification:
			&&& Clothing includes body modifications and tattoos, cosplay, media references, etc.
			&&& Discussion on video games, internet culture, etc.
			&&& Conventions allow for physical sharing of interests
				
		&& Physical games such as DnD, MtG, board games, etc.
		&& Multiplayer gameplay such as LAN parties, MMORPGs, eSports
		&& Specific interest meetups
			
		&& Values and aspirations:
			&&& Intelligence and intellectual pursuits
			&&& Gameplay skill
			&&& Up-to-date regarding nerdy topics
			&&& Employment as a software or game developer
			&&& Success in participatory gameplay
		&& New subculture members sometimes rejected by 'veterans' due to bitterness prior to nerd acceptance
			
		&& Threats - e.g. Gamergate: Scandal in which a female game developer was accused of manipulating others to become more powerful/famous
		&& Popular culture:
			&&& Many are upset with how people who once marginalized them now glorify them
			&&& Threats by cultural appropriation:
				&&&& Breaks the boundary between the gamers' world and the mainstream world
				&&&& E.g. Mass-marketed merchandise, popularity of widely-palatable non-casual games, etc.

\end{easylist}
\subsection{Streaming Gameplay}
	\label{subsec:participatory-gaming:streaming-gameplay}
\begin{easylist}

	& Callois described how mimicry vs agency, fantasy fulfillment, and looking up to a skilled person create an interest in watching others play games

	& eSports:
		&& Builds on this shared interest
		&& Changed from amateur to highly professional
			&&& E.g. Some universities offer scholarships to professional gamers
		&& Most popular game genres: MOBA, FPS, RTS, fighting
		&& Statistics:
			&&& League of Legends: 1300 tournaments
			&&& DOTA 2: 300 tournaments
			&&& StarCraft: 2200 tournaments
		&& Celebrities/online personalities:
			&&& Defined by low production values and a niche audience
			&&& Often humorous
			&&& Empowered by social video sites such as YouTube
			&&& Traditional media is challenged and threatened by this because they cannot control it or its growth
			&&& E.g. PewDiePie:
				&&&& Millions of subscribers
				&&&& Popularized games such as Flappy Bird and Goat Simulator simply by playing them in his videos						
						
	& Twitch TV: Website where people can stream live gameplay
		&& Fundamentally a second screen experience; no need to actively play to interact
		&& Fourth largest provider of online video behind Netflix, YouTube, Hulu
		&& 100 million unique users per month
		&& Purchased by Amazon for \$970 million
		&& Shares best practices for gamers and content creators
		&& Game industry is attempting to appeal to the same audience via:
			&&& Game optimization
			&&& Live events
			&&& Sponsorships
			&&& Celebrity engagement
			&&& Spectator modes
			&&& Contrasted with Bell Media in Canada, which is raising TV fees, blocking content, violating net neutrality, and fighting Netflix
		&& Raises questions about how to value this media, how to deal with legal/copyright issues, what constitutes as intellectual property, etc.

\end{easylist}
\clearpage
	%
% IAT 210: Introduction to Game Studies - A Course Overview
% Designing Games
%
% Author: Jeffrey Leung
%

\section{Designing Games}
	\label{sec:designing-games}
\subsection{General Game Design}
	\label{subsec:designing-games:general-game-design}
\begin{easylist}

	& Game mechanics:
		&& Add verbs/actions to increase the number of mechanical interactions
			&&& Allow actions to act on more objects
		&& Add nouns/objects
	& Ensure flow:
		&& Create small, achievable goals and manageable challenges
		&& Write clear rules
		&& Give immediate feedback
		
	& Allow game goals to be achieved through multiple gameplay gestalts (see subsection~\ref{subsec:the-players-perspective:gameplay-gestalts})
	
	& Balance:
		&& Cake-cutting example: Two people share a slice of cake. One person cuts the slice in half, the other person chooses who gets which slice.
			&&& Cutter has an incentive to cut the slice as fairly as possible
			&&& Ensures fairness and does not enable cheating

\end{easylist}
\subsection{Designing Rules}
	\label{subsec:designing-games:designing-rules}
\begin{easylist}
	
	& Designing rules:
		&& Designed iteratively with trial and error
		&& Order:
			&&& Create operational rules
			&&& Create foundational rules and ensure game balance
			&&& Use playtesting and iteration to cover every circumstance possible
			&&& Create written rules

\end{easylist}
\clearpage
	%
% IAT 210: Introduction to Game Studies - A Course Overview
% Reasons for Game Design
%
% Author: Jeffrey Leung
%

\section{Reasons for Game Design}
	\label{sec:reasons-for-game-design}
\begin{easylist}
				
	& Schell's 4 types of game designers:
		&& \emph{Persuasives:} Game designers who create games to fulfill a third-party goal
			&&& See \emph{Gamification and Motivational Design}, subsection~\ref{subsec:reasons-for-game-design:gamification-and-motivational-design}
		&& \emph{Humanitarians:} Game designers who design games for a greater good
			&&& See \emph{The Impact of Games on the World}, subsection~\ref{subsec:reasons-for-game-design:the-impact-of-games-on-the-world}
		&& \emph{Fulfillers:} Game designers who create games for gamers
		&& \emph{Artists/Indies:} Game designers who create games as a form of artistic expression
			&&& See \emph{Indie Games}, subsection~\ref{subsec:types-of-games:indie-games}

\end{easylist}
\subsection{Gamification and Motivational Design}
	\label{subsec:reasons-for-game-design:gamification-and-motivational-design}
\begin{easylist}

	& \emph{Gamification:} Application of game design conventions such as competitions and rewards to non-game activities
		&& Adds fun and engagement
		&& Wide range of potential applications (e.g. fitness, personal finances)
		&& Most people and hardcore gamers want different things
		&& Is enabled by:
			&&& A cultural framework which is friendly towards games
			&&& More people playing more games on an increasing range of devices
			&&& Tracking technology, web analytics, cultural momentum
		&& In today's world, gamification:
			&&& Has shifted the design of experiences towards pleasure
			&&& New developments engage with the user, are quicker to market, are cheaper to create, etc.
			&&& Is mostly driven by marketing teams
			&&& Relies on extrinsic motivators and non-fun ideas such as savings and status
		&& Somewhat controversial
			&&& Counter-argument - Ian Bogost's paper on \emph{Schell Games}:
				&&&& Refers to such experiences as exploitation-ware
				&&&& Questions whether measurable outcomes matter more than true player motivation
				&&&& Questions whether it is okay for incentives to muddy meaningful player agency
				
	& \emph{Motivational design:} Game design around giving the player a reason to play, which benefis both the player and designer
		&& Coined by Sebastian Deterding
			&&& Understand the difference between intrinsic and extrinsic motivation when we structure experiences to engage
			&&& Internalization can occur through the application of positive reinforcement; externalization can happen when rewards are applied heavily to an experience
			&&& Problem occurs when extrinsic factors are applied without taking into account the user questioning benefits for themselves
				
	& \emph{Self-determination:} Why people choose what they choose, when they are free to choose
		&& Field of psychology that has only recently been considered empirically valid
		&& Explains the three universal needs of a gamified experience:
			&&& \emph{Confidence:} Mastery of a pursuit
				&&&& See \emph{agon}, subsection~\ref{subsec:types-of-games:types-of-play}
			&&& \emph{Autonomy:} Control and self-direction
				&&&& See \emph{agency}, subsection~\ref{subsec:properties-of-games:attributes-of-games}
			&&& \emph{Relatedness:} Positive social environment and/or motivational climate
				&&& E.g. Twitch, nerd culture, fitness clubs
	& Pursuit of pleasure:
		&& Some play games to seek pleasure; some play games as social obligations (e.g. FarmVille, grinding)
		&& Some games shift motivations from pleasure-seeking to another motivator
		&& See \emph{14 Forms of Fun}, subsection~\ref{subsec:properties-of-games:attributes-of-games}

	& Motivationally designed experiences must:
		&& Hold strong appeal from the pleasure centre
		&& Engage with the intended user to encourage return
		&& Require minimal effort to play
		&& Disallow gaming of the system
		&& Not be embarrassing to peers
		&& Questions to ask about the application of gamification:
			&&& How to define the form of fun for an experience
			&&& Whether the intended users will enjoy the experience
			&&& How to ensure traits become more and more prevalent through repetition
			&&& How to create the right mix between intrinsic and extrinsic rewards
			&&& Whether it reinforces the business goal

\end{easylist}
\subsection{The Impact of Games on the World}
	\label{subsec:reasons-for-game-design:the-impact-of-games-on-the-world}
\begin{easylist}

	& Some believe games can create a better world, partially due to the proliferation of games in society
	& Some believe that solving imaginary problems in virtual worlds is mostly unproductive
	
	& Jane McGonigal:
		&& Game designer and theorist
		&& Believes that intrinsically motivated gameplay can lead to collective benefits for society
		&& Gamers are learning:
			&&& \emph{Urgent optimism:} Ability to act immediately to overcome a problem
			&&& \emph{Epic meaning:} Wanting to be a part of a larger purpose
			&&& Blissful productivity
			&&& Social skills
		&& Created a game named SuperBetter:
			&&& Adapts the challenge/reward structure of games
			&&& Helps players solve their own real-life problems such as obesity, depression, getting a job, asthma, etc.
			
	& Challenge is to find a way to use games to harness creative and productive energy
	& See \emph{Serious Games}, subsection~\ref{subsec:types-of-games:serious-games}

\end{easylist}
\clearpage
	%
% IAT 210: Introduction to Game Studies - A Course Overview
% Business
%
% Author: Jeffrey Leung
%

% Date: 2015-07-15

\section{Business}
	\label{sec:business}
\subsection{Project and Company Funding}
	\label{subsec:business:project-and-company-funding}
\begin{easylist}

	& Timing is important when it comes to fundraising

	& Types of investment:
		&& \emph{Dumb money:} Money donated with hidden conditions
			&&& E.g. Advertising for a company, changing content away from controversial subjects
		&& \emph{Silent money:} Money donated without conditions
		&& \emph{Smart money:} Money invested by those who are well-informed and/or experienced in the industry

	& Types of project funding:
		&& Self-funding through day job and personal wealth
			&&& Often not enough
		&& Prizes and contests (only when desperate)
		&& Money from family/friends
			&&& Will quickly dry up
		&& Crowdfunding through networks such as Kickstarter
		&& Government programs
	& Types of company funding:
		&& \emph{Incubator:} Company which provides management training, guidance, office space, and other services to startup companies
			&&& Government-funded
			&&& Receive no equity
			&&& May focus on biotechnology, medical technology
		&& \emph{Accelerator:} Cohort program which provides mentorship and education, with the goal of having the startups present their product to investors after a fixed period of time
			&&& May provide a physical workspace
			&&& Publicly or privately funded
			&&& May focus on a wide range of industries
		&& Angel investor: Entity who invests significantly in a venture in exchange for ownership equity
			&&& Difficult to find
			&&& Best type of investment to have
		&& Venture capitalist: Entity who provides money to new and emerging companies
		&& Traditional banking is no longer an option

\end{easylist}
\subsection{Business Models of Game Types}
	\label{subsec:business:business-models-of-game-types}
\begin{easylist}

	& Casual games:
		&& Lower costs spur early development
		&& Initial organic audience growth
		&& Greater attention leads to monetization
		&& Creates early returns due to easy scaling
		&& E.g. Zynga, Big Fish
		
	& Indie games:
		&& For an overview of indie games, see subsection~\ref{subsec:types-of-games:indie-games}
		&& Not always successful
		&& Relies on discoverability by the intended audience
		&& Compared to the mainstream game industry:
			&&& Less hit-driven
			&&& Small successes/failures are more relevant
		&& Developers need to be smart business people and marketers
		&& Raising money:
			&&& Tradeoff between time spent:
				&&&& Raising money
				&&&& Developing the game
			&&& `Raise only what you need, but raise enough'

\end{easylist}
\subsection{The Company}
	\label{subsec:business:the-company}
\begin{easylist}
		
	& Industries are discovering that making a great business is different from making a good game

	& \emph{Unicorn company:} Ideal company which utilizes a small team and big ideas to quickly make a continuously profitable product/line
		&& `Jewels' of Silicon Valley
			&&& Rarely works elsewhere; some of its properties don't create much traction
				&&&& E.g. Minimum Viable Product (MVP): Idea of creating a product which has the minimum required features to fulfil its purpose, in order to minimize costs
		&& Can hit a pre-public valuation of \$1 billion
		&& Nearly impossible to find
		&& Increasing numbers
			&&& Causing worries about the tech bubble popping
		&& E.g. Hootsuite, Slack
		
	& \emph{Rhino company:} Company which is well-rounded, realistic, and works through problems and failures at a slower, more reasonable pace
		&& Ugly, bulky, but have staying power
		&& Has more chances to pivot (change focus/direction) to find a product-market fit
		&& Being in a secondary market means a lower run-rate (spending over time)
		
	& `Think like a unicorn, work like a rhino'
		&& Starting a new venture is notoriously difficult
		&& Failures do not mean you cannot succeed
		
\end{easylist}
\clearpage
	%
% IAT 210: Introduction to Game Studies - A Course Overview
% The Audience
%
% Author: Jeffrey Leung
%

\section{The Audience}
	\label{sec:the-audience}
\subsection{Game Players}
	\label{subsec:the-audience:game-players}
\begin{easylist}

	& Stereotypically anti-social teenagers
	& Large amount of society
	& Ongoing feedback loop created by new platforms and rapid data transfer exists between:
		&& Businesses
		&& Game designers
		&& The audience
		
\end{easylist}
\subsection{Hardcore Gamers}
	\label{subsec:the-audience:hardcore-gamers}
\begin{easylist}

	& Consumers of a wide range of games
	& Can be competitive
	& Participate in game culture beyond the game itself
	
\end{easylist}
\subsection{Casual Gamers}
	\label{subsec:the-audience:casual-gamers}
\begin{easylist}

	& Limited interest in and time for video games
	& Do not usually self-identify as gamers
	& May play games types such as:
		&& Puzzle
		&& Hidden object
		&& Arcade
		&& Card/casino games
	& May play games on non-game devices
	& Requires games which:
		&& Are easy to learn
		&& Are quick to play
		&& Are easily accessible/convenient
		&& Have either no goal or easily achieveable goals
		&& Have a flat difficulty curve
	& Demographics: Older and more female than other types of gamers

\end{easylist}
\subsection{Player Personae}
	\label{subsec:the-audience:player-personae}
\begin{easylist}

	& Developers must:
		&& Understand their audience
			&&& \emph{Psychographic approach:} Analysis of interests, attitudes, and opinions
		&& Avoid self-referential design (designing with their own ideals in mind rather than the audience's ideals)
		
	& \emph{Persona:} Fictional depiction of the psychology and lifestyle of an ideal player
		&& Created by developers to target the game towards a certain audience
		&& Includes:
			&&& A picture
			&&& Personal history and details
			&&& Aspirations
			&&& Desires
			&&& Goals
			&&& Use-case scenario with the game
		&& Begins with a notion of the audience, and improved by data
		&& Different from a focus group

\end{easylist}
\clearpage
	%
% IAT 210: Introduction to Game Studies - A Course Overview
% The Player's Perspective
%
% Author: Jeffrey Leung
%

\section{The Player's Perspective}
	\label{sec:the-players-perspective}
\begin{easylist}
	
	& \emph{Natural advantage:} A skill or ability not directly related to the game which gives a player an upper-hand
		&& Unrelated to:
			&&& Knowledge of the game
			&&& Familiarity of the game's strategy or gameplay	
	
\end{easylist}
\subsection{Choices and Payoffs}
	\label{subsec:the-players-perspective:choices-and-payoffs}
\begin{easylist}

	& Choices in games can be modeled through payoff matrices
		&& E.g. Prisoner's Dilemma:
		\begin{quote}
		Two criminals are apprehended for a crime and are pressured to confess in isolation. If a prisoner defects, he is promised an easier sentence. If they cooperate by maintaining silence, they will both receive a heavier sentence.
		\end{quote}
		&& Payoffs:
			&&& \emph{Temptation payoff:} Winning at the other person's expense
			&&& \emph{Reward payoff:} Both players win by cooperating, but very little
			&&& \emph{Punishment payoff:} Both players lose for trying to cheat each other
			&&& \emph{Sucker payoff:} Losing due to trying to do the right thing, and the other person trying to cheat you

\end{easylist}
\subsection{Gameplay Gestalts}
	\label{subsec:the-players-perspective:gameplay-gestalts}
\begin{easylist}

	& \emph{Gameplay gestalt:} Personal strategy/pattern of actions allowing players to progress
		&& \emph{Emergent actions:} Subset of all possible actions driven by unique player strategies and created by the available mechanics
		&& Many games have numerous emergent patterns which appeal to different player psychologies
		&& May use exploits in a game
		
\end{easylist}
\subsection{Reward and Addiction}
	\label{subsec:the-players-perspective:reward-and-addiction}
\begin{easylist}

	& \emph{Reward:} Positive reinforcement for a goal-oriented behaviour
		&& Biologically modulated by dopamine
		&& In games:
			&&& Common; easy to earn
			&&& Short-term
			&&& Allows social comparison
			&&& E.g. Achievements, leveling up, daily quests, unlockable content, rare items and collections
			&&& 4 main types of reward-oriented behaviour:
				&&&& Exploring
				&&&& Collecting
				&&&& Achieving
				&&&& Killing
	& \emph{Addiction:} Excessive or compulsive repetition of an activity
		&& Can be created through high-frequency, low-effort rewards
		&& Directly linked to behaviours such as:
			&&& Craving
			&&& Tolerance
			&&& Withdrawal
			&&& Loss of control
			&&& Neglect of other activities
		&& Consequences of MMO play:
			&&& Negative:
				&&&& Lower psychological well-being
				&&&& Worse life outcomes
				&&&& Lower self-esteem
				&&&& Greater aggressiveness
			&&& Positive:
				&&&& In-game friendships and connections
				&&&& Teamwork
				&&&& Group identity
				&&&& Social interaction and immersion
		&& Treatment:
			&&& Online support forums
			&&& Therapy
			&&& Medication

\end{easylist}
\clearpage
	
\end{document}