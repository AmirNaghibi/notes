%
% IAT 267: Introduction to Technological Systems - A Course Overview
% Section: Technological Systems
%
% Author: Jeffrey Leung
%

\section{Technological Systems}
	\label{sec:technological-systems}
\subsection{Concepts and Components}
	\label{subsec:technological-systems:concepts-and-components}
\begin{easylist}

	& Iterative design:
		&& General process:
			&&& Imagine
			&&& Design
			&&& Build
			&&& Analyze product
		&& Loop of design and redesign is repeated to improve upon the product

	& Trade-offs:
		&& Due to multiple interacting domains and subsystems (e.g. mechanical system and code system), solving one problem may cause another
		&& \emph{Space/time trade-off:} Compression of images results in reduced transmission time/costs, lower quality, increased time necessary to compress/decompress the file
		&& There is no correct solution; all solutions come with benefits and drawbacks

	& Real-world constraints:
		&& Design of a system involves multiple parts and domains
		&& E.g. Sorting physical marbles into different bins will have the following issues:
			&&& Sensor system which differentiates marbles interacting with the mechanical system which moves them
			&&& Transportation of marbles:
				&&&& Regardless of mass/volume
				&&&& From one location to the correct bin

	& Feedback:
		&& \emph{Negative feedback:} Reaction based on a tendency to return to the original state and stabilize the system
			&&& E.g. When placing a ball at the bottom of a bowl, moving the ball will cause it to roll back to the bottom
			&&& E.g. When using a spring, applying and releasing pressure on the spring will cause it to uncompress to its original shape
		&& \emph{Positive feedback:} Reaction based on a tendency to change to another state and destabilize the system
			&&& E.g. When placing a ball at the top of a hill, moving the ball will cause it to roll away from the top of the hill
			&&& E.g. In a speaker and microphone system, if the microphone picks up sound from the speaker, the sound will be amplified in a loop and the sound will continually become louder

	& Complexity management:
		&& \emph{Abstraction:} Hiding a complex system behind a simple interface
			&&& E.g. Programmed devices can be used effectively without knowledge of how they are designed
		&& \emph{Modularity:} Composition of a system from several independent, reusable partitions
			&&& E.g. A circuit which detects temperature and turns it into a value can be used in a car system, a thermostat, a weather balloon, etc.

\end{easylist}
\subsection{Categories}
	\label{subsec:technological-systems:categories}
\begin{easylist}

	& \emph{Digital:} System with a limited amount of states
		&& E.g. Whether or not brakes are activated
	& \emph{Analog:} System with a continuous range of possible states
		&& E.g. Speed of a car

	& \emph{Series:} System in which events happen in order
	& \emph{Parallel:} System in which events happen concurrently

\end{easylist}
\subsection{Interaction}
	\label{subsec:technological-systems:interaction}
\begin{easylist}

	& Interaction: Communication between multiple entities

	& In a technological system:
		&& Input:
			&&& Often requires less energy than output
			&&& See section \hyperref[sec:sensors]{Sensors}
		&& Processing of information
			&&& Requires programming
		&& Output
			&&& Often requires more energy than input
			&&& See section \hyperref[sec:actuators]{Actuators}

	& \emph{Transduction:} Transformation of energy from one form to another
		&& Necessary for interaction in technological systems
		&& E.g. Heat energy is transformed into sensor signals for a thermostat to interpret

		&& \emph{Transducer:} Device which changes energy from physical to electrical, or vice versa
			&&& Generally, input transducers require less energy than output transducers
			&&& Input transducers:
				&&&& E.g. Switches, levers, sliders
				&&&& See section \hyperref[sec:sensors]{Sensors}
			&&& Output transducer:
				&&&& E.g. Motors, servos, speakers, lights
				&&&& See section \hyperref[sec:actuators]{Actuators}

	& Communication between devices:
		&& \emph{Serial communication:} Transmission and receiving of data at a set rate of digital pulses
			&&& Data bits are sent in serial - i.e. one after another
		&& \emph{Communication protocol:} Standard of data transmission and reception shared by two communicating devices
			&&& Specifies characteristics such as physical configuration, timing, electrical connection, and package size
			&&& \emph{Data rate:} Frequency of data transmission
				&&&& Unit of pulses per second: Baud (Bd)
			&&& \emph{Asynchronous communication:} Communication between devices which run on their own independent clocks
			&&& Electrical connection is the interpretation of values (e.g. 0V maps to 0, 5V maps to 1)
			&&& Package size is the grouping of values (e.g. discrete values of 8 bits)
		&& See section \hyperref[sec:networking]{Networking}

\end{easylist}
\subsection{Haptic Technology}
	\label{subsec:technological-systems:haptic-technology}
\begin{easylist}

	& \emph{Haptic technology:} Technology which simulates the sensation of touch interaction using vibration, movement, or temperature

	& Possible ways of sensing input:
		&& Capacitance
		&& Force
		&& Flex
		&& Pressure
		&& Heat (using a thermistor)

\end{easylist}
\subsection{Building a System}
	\label{subsec:technological-systems:building-a-system}
\begin{easylist}

	& Process:
		&& Describe the end product from the user's point of view (i.e. end-user sensory input/output)
		&& Separate tasks into input, processing, and output
		&& Determine each input/output as \hyperref[subsec:technological-systems:categories]{digital or analog} (and, therefore, what kind of transducers are necessary)
		&& Determine whether each event should be processed in serial or parallel

\end{easylist}
\clearpage
