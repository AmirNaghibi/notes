%
% STAT 100: Chance and Data Analysis - A Course Overview
% Section: Analysis of Population Proportions
%
% Author: Jeffrey Leung
%

\section{Analysis of Population Proportions}
	\label{sec:analysis-of-population-proportions}
\subsection{Margin of Error and Confidence Interval}
	\label{subsec:analysis-of-population-proportions:margin-of-error-and-confidence-interval}
\begin{easylist}

	& \emph{Parameter:} Value which summarizes population data
		&& Calculation requires collection of data from the entire population (i.e. see \emph{census}, subsection~\ref{subsec:data-collection:methods})
		&& Estimation is often calculated from a sample statistic
			&&& E.g. If 33\% of a random sample of Canadian adults support the Conservative Party, then the proportion of all Canadian adults who support the Conservative Party is estimated to be 33\%.
		&& Denoted by $p$
			&&& Sample proportion/statistic is denoted by $\hat{p}$
	
	\medskip
	& \emph{Margin of error:} Percentage value of the uncertainty of an estimated population proportion
		&& Calculation (for a 95\% confidence level):
		\begin{math}
			Margin\ of\ error = \frac{1}{\sqrt{sample\ size}}
		\end{math}
			&&& Unit: Percentage
		&& Valid only for a random sample
		&& Dependent on a confidence level
			&&& \emph{Confidence level:} Degree of certainty of the accuracy of a population proportion estimate
				&&&& Unit: Percentage
				&&&& Often 95\% (expressed as a fraction; 19 times out of 20 = $\frac{19}{20}$)
		&& Interpretation (with the confidence level): \smallskip \\
		If many random samples of <sample size, subject> of the population are taken and the sample proportion of <statistic> is calculated for each sample, <confidence level percentage> of the sample proportions will be within $\pm$ <margin of error percentage> of the population proportion.
		
		&& E.g. ``A probability sample of this design and sample size would carry a margin of error in the range of $\pm 1.2\%$, 19 times out of 20.'' \smallskip \\
		Margin of error: $\pm 1.2\%$ \\
		Confidence level: $\frac{19}{20} = 95\%$
		&& E.g. Given a study of Canadian adults who support the Conservatives with a random sample of 6005 Canadian adults and a margin of error of $\pm$ 1.2\%, 19 times out of 20: \smallskip \\
		If many random samples of 6005 Canadian adults of the population are taken and the sample proportion of Canadian adults who support the Conservatives is calculated for each sample, 95\% of the sample proportions will be within $\pm$ 1.2\% of the population proportion.
	
	\medskip
	& \emph{Confidence interval:} Set of values which the population proportion is within
		&& Calculation:
		\begin{math}
			Confidence\ interval = sample\ proportion \pm margin\ of\ error
		\end{math}
			&&& Unit: Percentage range
		&& Interpretation: \smallskip \\
		Using the sample data, we are <confidence level percentage> confident that the population proportion of <statistic> is between <confidence interval lower bound percentage> and <confidence interval upper bound percentage>.
			&&& Always specify the population proportion
		&& E.g. Given the sample proportion of Canadian adults who will vote for the Conservatives as 33\% for 6005 subjects, the confidence interval is: \\
		\begin{math}
			Sample\ proportion \pm \frac{1}{\sqrt{sample\ size}}
			= 33\% \pm \frac{1}{\sqrt{6005}}
			= 33\% \pm 1.2904...\%
			\approx (31.7\%, 34.3\%)
		\end{math} \\
		Using the sample data, we are 95\% confident that the population proportion of Canadian adults who support the Conservative Party is between 31.7\% and 34.3\%.
		
		&& Analyzing the confidence interval: Given a value, check whether or not all values of the confidence interval satisfy the condition
			&&& E.g. Given the confidence interval for the population proportion of Canadian adults who support the Conservative Party to be (31.7\%, 34.3\%), can you conclude that more than 30\% of all Canadian adults support the Conservative Party? \smallskip \\
			Yes; all values in the confidence interval are greater than 30\%.
			&&& E.g. Given the confidence interval for the population proportion of Canadian adults who support the Conservative Party to be (31.7\%, 34.3\%), can you conclude that more than 34\% of all Canadian adults support the Conservative Party? \smallskip \\
			No; there exist values in the confidence interval which are less than 34\%.
			
\end{easylist}
\subsection{Bias and Variability}
	\label{subsec:analysis-of-population-proportions:bias-and-variability}
\begin{easylist}

	& \emph{Bias:} Consistent under-estimation or over-estimation of results
		&& Can be reduced by:
			&&& Avoiding \hyperref[subsec:data-collection:sampling-errors]{non-sampling errors}
			&&& Ensuring fair representation of the population
			&&& Using random sampling
		&& Increasing sample size does not reduce bias
			
	& \emph{Sampling variability:} Degree of variability between random samples
		&& Quantified by the \hyperref[subsec:analysis-of-population-proportions:margin-of-error-and-confidence-interval]{margin of error}
		&& Can be reduced by:
			&&& Increasing sample size
			
\end{easylist}
\subsection{Hypothesis Testing for Population Proportions}  %TODO move this to a different file
	\label{subsec:analysis-of-population-proportions:hypothesis-testing-for-population-proportions}
\subsubsection{Introduction}
	\label{subsubsec:analysis-of-population-proportions:hypothesis-testing-for-population-proportions:introduction}
\begin{easylist}

	& \emph{Hypothesis test:} Calculation which determines whether a claim/research hypothesis is supported by evidence

	& \emph{Null hypothesis (H\textsubscript{o}):} Statement that a population proportion is equal to a given value (which may be another population proportion)
		&& Assumed to be possible until contradicting evidence is found
	& \emph{Alternative hypothesis (H\textsubscript{a}):} Statement that a population proportion is less than, not equal to, or greater than a given value (which may be another population proportion)
	& E.g. The sample proportion of Canadian adults who want to legalize marijuana was found to be 59\%. Test whether or not the population proportion of Canadian adults who want to legalize marijuana is greater than 50\%. \\
	H\textsubscript{o}: The population proportion of Canadian adults who want to legalize marijuana is equal to 50\%. \\
	H\textsubscript{a}: The population proportion of Canadian adults who want to legalize marijuana is greater than 50\%.
	
	\medskip
	& \emph{Test statistic:} Standardized value representing a numerical summary of sample data 
		&& Calculated from sample data; analyzed to determine the p-value
		&& E.g. Z-statistic, t-statistic, chi-square statistic
		&& \emph{Z-statistic:} Test statistic which can be calculated to determine whether an inequal relationship between population proportions exists
			&&& Formula for one population proportion compared against a given percentage: \smallskip \\
			\begin{displaymath}
				z-statistic = 
				\frac
				{
					\hat{p} - p_{0}
				}
				{
					\sqrt
					{
						\frac
						{
							p_{0} \cdot (1 - p_{0})
						}
						{
							n
						}
					}
				}
			\end{displaymath}
			\Deactivate
			\begin{center}
				\begin{tabular}{ l r @{ = } l }
					where & $\hat{p}$ & sample proportion \\
					& $p_{0}$ & population proportion if H\textsubscript{0} is true \\
					& $n$ & sample size
				\end{tabular}
			\end{center}
			\Activate
			
			\medskip
			&&& Formula for two population proportions compared against each other: \medskip \\
			\begin{displaymath}
				z-statistic =
				\frac
				{
					\hat{p_{1}} - \hat{p_{2}}
				}
				{
					\sqrt
					{
						\hat{p} \cdot (1-\hat{p}) \cdot
						(
							\frac{1}{n_{1}} + \frac{1}{n_{2}}
						)
					}
				}
			\end{displaymath}
			\Deactivate
			\begin{center}
				\begin{tabular}{ l r @{ = } l }
					where & $\hat{p_{x}}$ & sample proportion of the $x$\textsuperscript{th} set of data \\
					& $\hat{p}$ & combined sample proportion \\
					& $n_{x}$ & sample size of the $x$\textsuperscript{th} set of data
				\end{tabular}
			\end{center}
			\Activate
			
				&&&& Combined sample proportion is the sum of the number of subjects satisfying the condition of sample dataset 1 and the number of subjects satisfying the condition of sample dataset 2, divided by the sum of the number of subjects of each dataset
			
		&& \emph{Chi-square statistic:} %TODO
	
	\medskip
	& \emph{Probability value (p-value):} Probability that a given result is obtained through chance, calculated from sample data
		&& Unit: Percentage
		&& Interpretation: If many random samples of the given sample size and population are chosen and the sample proportion in question is calculated, then the p-value represents the percentage of the sample populations which would support the alternative hypothesis.
			&&& E.g. Given that 50\% of all Canadian adults support the legalization of marijuana, the probability of calculating a sample proportion of 59\% or higher through random sampling is equal to the p-value (0.62\%).
		&& The lesser the p-value, the greater the evidence for the alternative hypothesis
		&& Calculated from a test statistic; compared against the significance level to determine the amount of evidence for the alternative hypothesis
		&& For the interpretation of the p-value compared against the significance level, see \hyperref[subsec:analysis-of-population-proportions:statistical-significance]{statistical significance}
		&& For the interpretation of the magnitude, see table~\ref{tab:p-value-magnitude-chart}
		
		\Deactivate
		\begin{table}[!htb]
			\centering
			\caption{P-Value Magnitude Chart}
			\label{tab:p-value-magnitude-chart}
			\begin{tabular}{ r c l l }
				\multicolumn{3}{ c }{P-value} & Strength of Evidence to Support H\textsubscript{a} \\
				 10\% < & p-value & & No evidence \\
				  5\% < & p-value & $\leq$  10\% & Weak evidence \\
				  1\% < & p-value & $\leq$   5\% & Some evidence \\
				0.1\% < & p-value & $\leq$   1\% & Strong evidence \\
				        & p-value & $\leq$ 0.1\% & Very strong evidence
			\end{tabular}
		\end{table}
		\Activate
		
		&& P-value of a z-statistic: Area under the standard normal distribution where z-statistic equals the z-score (see %\ref standard normal table
			&&& If the alternative hypothesis is `less than', the p-value is the area to the left of the z-statistic
			&&& If the alternative hypothesis is `not equal to', the p-value is the area to the left of the negative absolute value of the z-statistic an the area to the right of the positive absolute value of the z-statistic
			&&& If the alternative hypothesis is `greater than', the p-value is the area to the right of the z-statistic
		&& P-value of a chi-square statistic: %TODO
		
	& \emph{Significance level:} %TODO
			
	& General process:
		&& Find the test statistic using the sample data
		&& Find the p-value using the test statistic
		&& Compare the p-value to the significance level
			&&& Conclusion: ``Since the p-value (<p-value>) is <less than/greater than> the significance level (<significance level>), we <do not> reject the null hypothesis. There is <sufficient/insufficient> evidence to conclude that <alternative hypothesis>.''

\end{easylist}
\subsection{One Population Proportion (Z-Statistic)}
	\label{subsec:analysis-of-population-proportions:one-population-proportion-z-statistic}
\begin{easylist}
	
	& 
	& E.g. %TODO

\end{easylist}
\subsection{Two Population Proportions (Z-statistic)}
	\label{subsec:analysis-of-population-proportions:two-population-proportions-z-statistic}
\begin{easylist}

	& Process:
		&& Compute the proportion of subjects in each test group who satisfy the condition
		&& Compare the proportions using a bar graph
		&& Conclude which group has a greater/lesser proportion
		&& To compare population proportions, conduct a hypothesis test (see %TODO ref
			&&& Null hypothesis states that the population proportions are equal; alternative hypothesis states that the population proportions are unequal (less than, not equal to, or greater than)
			&&& Formula for the z-statistic for two population proportions:
		
	& E.g. Are the proportions...
	
\end{easylist}
\subsubsection{Confidence Interval}
	\label{subsubsec:analysis-of-population-proportions:two-population-proportions-z-statistic:confidence-interval}
\begin{easylist}

	& Formula (95\% confidence interval):
	\begin{math}
		(\hat{p_{1}} - \hat{p_{2}}) \pm 2 \cdot
		\sqrt
		{
			\frac
			{
				\hat{p_{1}} \cdot (1 - \hat{p_{1}})
			}{
				n_{1}
			}
			+ \frac
			{
				\hat{p_{2}} \cdot (1-\hat{p_{2}})
			}{
				n_{2}
			}
		}
	\end{math}
	
	& Example: [..] How does the proportion of people who have lung cancer...
	
	& There may be no difference between the proportion of students who use iPhones in UBC compared to the proportion of students who use iPhones in SFU.
	
\end{easylist}
\subsection{Multiple Population Proportions (Chi-Square Statistic)}
	\label{subsec:analysis-of-population-proportions:multiple-population-proportions-chi-square-statistic}
\begin{easylist}

	& The greater the difference, the greater the chi-square statistic
	& P-value always uses the right side of the normal distribution
	& Only concludes whether a relationship exists between two variables
	& Can compare many population proportions

\end{easylist}
\subsection{Errors}
	\label{subsec:analysis-of-population-proportions:errors}
\begin{easylist}

	& \emph{Type I error:} Rejection of H\textsubscript{o} from analysis of the sample data when H\textsubscript{o} is true
		&& I.e. False positive/confirmation of the alternative hypothesis; finding evidence where there is none
		&& May occur when H\textsubscript{o} is rejected
		&& E.g. Judging a person for a crime: \\
		H\textsubscript{o}: The person is not guilty. \\
		H\textsubscript{a}: The person is guilty. \\
		Truth: The person is not guilty (H\textsubscript{o} is true). \\
		Decision: The person is guilty (H\textsubscript{o} is rejected).
		&& Probability of its occurrence is directly proportional to the \hyperref[subsubsec:analysis-of-population-proportions:hypothesis-testing-for-population-proportions:introduction]{significance level}
			&&& Reducing significance level (and therefore, the probability of the type I error) increases the probability of the type II error
		
	& \emph{Type II error:} Failure to reject H\textsubscript{o} from analysis of the sample data when H\textsubscript{a} is true
		&& I.e. Failing to find evidence which exists
		&& May occur when H\textsubscript{o} is not rejected
		&& E.g. Judging a person for a crime: \\
		H\textsubscript{o}: The person is not guilty. \\
		H\textsubscript{a}: The person is guilty. \\
		Truth: The person is guilty (H\textsubscript{a} is true). \\
		Decision: The person is not guilty (H\textsubscript{o} is not rejected).
		&& Probability of its occurrence is inversely proportional to the \hyperref[subsubsec:analysis-of-population-proportions:hypothesis-testing-for-population-proportions:introduction]{significance level}
			&&& Increasing significance level (and therefore, the probability of the type II error) increases the probability of the type I error
			
	& E.g. A company will renew a contract with a radio station only if the station can find sufficient evidence to support that more than 20\% of the listeners have heard their ad. The station conducts a random survey of 400 people, 100 of which have heard the ad. \smallskip \\
	H\textsubscript{o}: 20\% of the listeners have heard the ad. \\
	H\textsubscript{a}: More than 20\% of the listeners have heard the ad. \smallskip \\
	A type I error will occur if 20\% of the listeners have heard the ad, but the sample data provides sufficient evidence to conclude that more than 20\% of the listeners have heard the ad. H\textsubscript{o} is true but rejected; H\textsubscript{a} is false but accepted. \\
	The possibility of this error can be reduced by decreasing the significance level. \smallskip \\
	A type II error will occur if more than 20\$ of the listeners have heard the ad, but the sample data does not provide sufficient evidence to reject the hypothesis that 20\% of the listeners have heard the ad. H\textsubscript{o} false but not rejected; H\textsubscript{a} is true but not affirmed. \\
	The possibility of this error can be reduced by increasing the significance level.
	
\end{easylist}
\subsection{Statistical Significance}
	\label{subsec:analysis-of-population-proportions:statistical-significance}
\begin{easylist}

	& P-value represents the probability that a given sample proportion is found, if the null hypothesis is correct, and therefore the probability of obtaining a difference of | sample proportion - test proportion | from the test proportion
	
	& \emph{Statistically significant:} Result which is unlikely to occur by chance
		&& Statistically significant result: P-value is less than the significance level
		&& Not statistically significant result: P-value is greater than the significance level
	
	& If a p-value is less than the significance level, then the difference between the sample proportion and the test proportion is statistically significant at the given significance level
	& If a p-value is greater than the significance level, then the difference between the sample proportion and the test proportion is not statistically significant at the given significance level


\end{easylist}
\clearpage